\documentclass{article}

\usepackage{amssymb, amsmath, amsfonts}

\usepackage{color, soul}

\title {Hw 7}
\date{}

\begin{document}

\maketitle

\textbf{1)}  We're given:  

\medskip

$\beta=(132)$

\medskip

$\alpha=(23)$

\medskip


$\gamma=(12)$

\medskip

$\delta=(123)$

\medskip

$\kappa=(13)$

\medskip

$H=H_{\epsilon}=\{\epsilon, \alpha \}$ where we have the $\epsilon$ is the identity and $\alpha=(23)$.

\medskip

$H=H_{\epsilon}=\{\epsilon, \alpha\}$


\medskip

\hspace{.11in} $=H_{\alpha}=\{\epsilon \circ \alpha, \alpha \circ \alpha \}$

\hspace{.48in} $=\{ \alpha, \epsilon \}$

\medskip

$H_{\beta}=\{\epsilon \circ \beta, \alpha \circ \beta \}$

\hspace{.18in} $=\{\beta, \gamma\}$

\medskip 

$H_{\gamma}=\{\epsilon \circ \gamma, \alpha \circ \gamma \}$

\hspace{.18in} $=\{\gamma, \beta\}$

\medskip

$H_{\delta}=\{\epsilon \circ \delta, \alpha \circ \delta \}$

\hspace{.18in} $=\{\delta, \kappa\}$

\medskip

$H_{\kappa}=\{\epsilon \circ \kappa, \alpha \circ \kappa \}$

\hspace{.18in} $=\{\kappa, \delta\}$

\medskip

With all this, we can say that 

\medskip

$H=H_{\epsilon}=H_{\alpha}=\{\epsilon, \alpha\}$

\medskip

$H_{\beta}=H_{\gamma}=\{\beta, \gamma\}$

\medskip

$H_{\delta}=H_{\kappa}=\{\delta, \kappa\}$

%%%%%%%%%%%%%%%%%%%%%%%%%%%%%%%%%%%%%%
\newpage

\textbf{2)} We have $G=\mathbb{Z}_{15}$ and going to list cosets of $H=<5>$. We know that $\mathbb{Z}_{15}=\{1, 2, 3,.......14\}$ and $<\mathbb{Z} ,+>$.

\medskip

The cosets of H is H=H+0, H+1, H+2, H+3, H+4. Note that H+the numbers=the numbers+H is abelian. 

\medskip

Let's check:

\medskip

$H=\{0, 5, 10\}$

\medskip

$H+1=\{0+1, 5+1, 10+1\}=\{1, 6, 11\}$

\medskip

$H+2=\{2, 7, 12\}$

\medskip

$H+3=\{3, 8, 13\}$

\medskip

$H+4=\{4, 9, 14\}$




%%%%%%%%%%%%%%%%%%%%%%%%%%%%%%%%%%%%%%
\newpage

\textbf{3)} Describing the cosets of subgrp $<3>$ in $\mathbb{Z}$. 

\medskip

First, we know that Z contains the negative and positive integers ad zero. With that idea, then $H=<3>=\{........-9,-6,-3,0,3,6,9.....\}$.
Lets try out some numbers:

\medskip

$H+1=\{.....-9+1, -6+1, -3+1, 0+1, 3+1, 6+1, 9+1.......\}$

\hspace{.4in}$=\{......-8, -7, -2, 1, 4, 7, 10....\}$

\medskip

$H+2=\{......-7, -4, -1, 2,  5, 8, 11....\}$

\medskip

As you can see, the elements show are either subtracted from 3 or added by 3.  If we did H+3, it would equal to H.

Check:

\medskip

$H+3=\{......-6, -3, 0, 3,  6,....\}$


\medskip


We can easily say that our subgroup H contains the elements of the group:

\medskip

$H=\{3n : n \in \mathbb{Z} \}$



%%%%%%%%%%%%%%%%%%%%%%%%%%%%%%%%%%%%%%
\newpage

\textbf{4)} Prove that $x^{n}=e$ for every x $\in$ G. First, note that the Lagrange's theorem says that H $\leq$ G and G is finite, where $|H|/|G|$.

\medskip 

Suppose H $\leq G, let H=<x^{m}>$ where m is the smallest positive integer and $G=<x>$ for x $\in G$.

\medskip

We have $x^{n}=e$, then by Division Algorithm

\medskip

$$n=mq+r$$

$$e=x^{n}=x^{mq+r}=x^{mq}x^{r}=(x^{m})^{q}x^{r}=x^{r}$$


Note that r=0 because m is the smallest positive integer s.t. $x^{m}=e$ and $r < m.$ 

\medskip

So 

$$n=mq+0$$
$$n=mq$$


so $$m|n \mapsto |H|/|G|$$





%%%%%%%%%%%%%%%%%%%%%%%%%%%%%%%%%%%%%%
\newpage

\textbf{5)}  Note that Homomorphism means that $f(ab)=f(a)f(b)$for all a,b $\in$ G. This idea is like isomorphism, excluding 1-1 and onto. 

\medskip

Since $D(\mathbb{R}$ is the additive group then let f,g $\in \mathbb{R}$ then 

\medskip

$\phi(f)=f'$

\medskip

$\phi(f+g)'=(f+g)'$

\medskip


 \hspace{.6in}$=(f)'+(g)'$
 
 \medskip

 
 \hspace{.6in}$=\phi(f')+\phi(g')$
 
 \medskip

Note that definition of kernel of f: kern f=$\{g \in G | f(g)=e_{H} \}$, so basically any element that would equal to 0. The kern($\phi$) would equal to any constant numbers in f and g because any constant numbers of derivatives is equal to 0. 


%%%%%%%%%%%%%%%%%%%%%%%%%%%%%%%%%%%%%%
\newpage

\textbf{6)}  Proof: Let A,B $\in$ G and G be the multiplicative group of all $2X2$ matrices over $\mathbb{R}$.

\medskip

We are given f(A)=det(A), then

\medskip

$f(A)=det(A)$


$f(AB)=det(AB)$


\hspace{.45in}$=det(A)det(B)$


\hspace{.45in}$=f(A)f(B) \surd$

\medskip

The kern(f) would equal to a zero in det(A) or in det(B) would give us zero when det(A) and det(B) are multiplied together.

%%%%%%%%%%%%%%%%%%%%%%%%%%%%%%%%%%%%%%
\newpage

\textbf{7)} Proof: Let a,b $\in$ G

\medskip

then (note: we're in the idea of composition)

\medskip

$g \circ f (ab)=g(f(ab))$

\medskip

\hspace{.55in}$=g(f(a)f(b))$

\medskip

\hspace{.55in}$=g(f(a))g(f(b))$

\medskip

\hspace{.55in}$=(g\circ f)(a)(g\circ f)(b) \surd$

\medskip

\medskip


The kernel of g $\circ$ f would equal to some number in g $\circ$ some number in f that will give us the identity. 





%%%%%%%%%%%%%%%%%%%%%%%%%%%%%%%%%%%%%%
\newpage

\textbf{8)} To prove that f(K) is a subgroup of H, we have to check if f(K) passes three test: non-empty, closure, and inverse.

\medskip

 
1) non-empty: since K is any subgroup of G (K$\leq G)$, there's an identity $(f(e_{G})=e_{H} )$, then f(K) is non-empty. 

\medskip

2) closure: Fix any x,y $\in$ K. Then f(x)=$e_{H}$=f(y).

so 

\medskip

$f(x)=f(x)$


$f(xy)=f(x)f(y)$


\hspace{.35in} $=e_{H}e_{H}$


\hspace{.35in} $=e_{H}$

\medskip

Yes, x,y $\in$ K.

\medskip

3) inverse: fix any x $\in$ K, so f(x)=$e_{H}$

\medskip

then 

$$f(x^{-1})=f(x)^{-1}=e_{H}^{-1}=e_{H}$$

\medskip

$\therefore$  f(K) is a subgroup of H.


\end{document}