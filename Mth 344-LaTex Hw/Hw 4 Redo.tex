\documentclass{article}

\usepackage{amssymb, amsmath, amsfonts}

\usepackage{color, soul}

\title{Hw 4 Redo}

\date{}

\begin{document}
\maketitle

\textbf{1)} 
We know that:

Redo: 1 , 4, 6, 9, 10

\medskip


$f=(\begin{smallmatrix} 
1 & 2 & 3 & 4 & 5 & 6 \\ 
6 & 1 & 3 & 5 & 4 &2
\end{smallmatrix}) $
				$ \mapsto f^-1=(\begin{smallmatrix} 
				1 & 2 & 3 & 4 & 5 & 6 \\ 
				2 & 6 & 3 & 5 & 4 &1
				\end{smallmatrix}) $ 			

\bigskip
 

%\smallskip

$ g=(\begin{smallmatrix} 
1 & 2 & 3 & 4 & 5 & 6 \\									
2 & 3 & 1 & 6 & 5 & 4
\end{smallmatrix}) $
				$ \mapsto g^-1=(\begin{smallmatrix} 
				1 & 2 & 3 & 4 & 5 & 6 \\									
				3 & 1 & 2 & 6 & 5 & 4
				\end{smallmatrix}) $
					
	
\bigskip

	



$h=(\begin{smallmatrix} 
1 & 2 & 3 & 4 & 5 & 6 \\
3 & 1 & 6 & 4 & 5 & 2
\end{smallmatrix}\ )$
					$\mapsto h^-1=(\begin{smallmatrix}
					1 & 2 & 3 & 4 & 5 & 6 \\
					2 & 6 & 1 & 4 & 5 & 3
					\end{smallmatrix})$	

\bigskip

\hspace{.2in}\textbf{a)} $g \circ h^{-1}$

\medskip

$g  \circ  h^{-1}=(\begin{smallmatrix}
1 & 2 & 3 & 4 & 5 & 6 \\	
3 & 4 & 2 & 6 & 5 & 1
\end{smallmatrix})$

\medskip

	Check: \begin{center}
				$g(h^{-1}(1))=g(2)=3 $ \\
				
				$g(h^{-1}(2))=g(6)=4 $ \\
									          $\cdot$ \\
									          $\cdot$ \\
									          $\cdot $\\
				$g(h^{-1}(6))=g(3)=1 $					
				
				\end{center}
\medskip

\hspace{.2in}\textbf{b)}

$h \circ g^-1 \circ f^-1=(\begin{smallmatrix}
1 & 2 & 3 & 4 & 5 & 6 \\	
3 & 4 & 1 & 5 & 2 & 6
\end{smallmatrix})$

\medskip

Check: \begin{center}
				$h(g^-1(f^-1(1)))=h(g^-1(2)) =h(1)=3 $ \\
				
				$h(g^-1(f^-1(2)))=h(g^-1(6))=h(4)=4 $ \\
									          $\cdot$ \\
									          $\cdot$ \\
									          $\cdot $\\
				$h(g^-1(f^-1(6)))=h(g^-1(1))=h(3)=6 $					
				
				\end{center}
%%%%%%%%%%%%%%%%%%%%%%%%%%%%%%%%%%%%%%%%%%%%%%%%%%%%
\newpage


\textbf{4} We have have:

\medskip

$f=(\begin{smallmatrix}
1 & 2 & 3 & 4 & 5 \\
2 & 1 & 3 & 4 & 5
\end{smallmatrix})$

\medskip

$g=(\begin{smallmatrix}
1 & 2 & 3 & 4 & 5 \\
1 & 2 & 4 & 5 & 3
\end{smallmatrix})$

\medskip

$h=(\begin{smallmatrix}
1 & 2 & 3 & 4 & 5 \\
1 & 2 & 3 & 4 & 5
\end{smallmatrix})=g^{2}$

\medskip

$l=(\begin{smallmatrix}
1 & 2 & 3 & 4 & 5 \\
2 & 1 & 5 & 3 & 4
\end{smallmatrix})$

\medskip


$f \circ g=(\begin{smallmatrix}
1 & 2 & 3 & 4 & 5 \\
2 & 1 & 4 & 5 & 3
\end{smallmatrix})=k$

\begin{table}[ht]
	\begin{tabular}{|c| |c| |c| |c| |c| c| |c|}

	\hline
	$\circ$ & $\epsilon$ & f & g & h & k & l \\
	
	\hline
	$\epsilon$ & \hl{$\epsilon$} & f & g & h & k & l \\
	
	\hline
	f & f & \hl{$\epsilon$} & k & l & g & h \\
	
	\hline
	g & g & k & h & \hl{$\epsilon$} & l & f \\
	
	\hline
	h & h & l & \hl{$\epsilon$} & g & f & k \\
	
	\hline
	k & k & g & l & f & h & \hl{$\epsilon$} \\
	
	\hline
	l & l & h & f & k & \hl{$\epsilon$} & g 
	
	\end{tabular}
\end{table}
	

%%%%%%%%%%%%%%%%%%%%%%%%%%%%%%%%%%%%%%%%%%%%%%%%%%%%
\newpage

\textbf{6)} We know that it is in $S_9$, so: 

\bigskip


\hspace{.3in}\textbf{a)}
$(145)(37)(682)$             $\mapsto (\begin{smallmatrix}
					1 & 2 & 3 & 4 & 5 & 6 & 7 & 8 & 9 \\
					4 & 6 & 7 & 5 & 1 & 8 & 3 & 2 & 9
					\end{smallmatrix})$ 
					
					\bigskip
					
\hspace{.3in}\textbf{b)}
$(17)(628)(9354)$             $\mapsto (\begin{smallmatrix}
					1 & 2 & 3 & 4 & 5 & 6 & 7 & 8 & 9\\
					7 & 8  & 5 & 9 & 4  & 2 & 1 & 6 & 3
					\end{smallmatrix})$ 
					
					\bigskip

\hspace{.3in}\textbf{c)}
$(71825)(36)(49)$             $\mapsto (\begin{smallmatrix}
					1 & 2 & 3 & 4 & 5 & 6 & 7 & 8 & 9 \\
					 8& 5 & 6 & 9 & 7 & 3 & 1 & 2 & 4
					\end{smallmatrix})$ 
					
					 \bigskip
\hspace{.3in}\textbf{d)}
$(12)(347)$            		 $\mapsto (\begin{smallmatrix}
					1 & 2 & 3 & 4 & 5 & 6 & 7 & 8 & 9\\
					2 & 1 & 4 & 7 & 5 & 6 & 3 & 8 & 9
					\end{smallmatrix})$ 
					
					 \bigskip

\hspace{.3in}\textbf{e)}
$(147)(1678)(74132)$     $\mapsto (\begin{smallmatrix}
					1 & 2 & 3 & 4 & 5 & 6 & 7 & 8 & 9 \\
					3 & 8 & 2 & 6 & 5 & 1 & 7 & 4 & 9
					\end{smallmatrix})$ 
					
					\bigskip
					
					
					
\hspace{.3in}\textbf{f)}
$(6148)(2345)(12493)$   $\mapsto (\begin{smallmatrix}
					1 & 2 & 3 & 4 & 5 & 6 & 7 & 8 & 9 \\
					3 & 5 & 4 & 9 & 2 & 1 & 7 & 6 & 8
					\end{smallmatrix})$
					
					



%%%%%%%%%%%%%%%%%%%%%%%%%%%%%%%%%%%%%%%%%%%%%%%%%%%%

\newpage



\textbf{9)} Even or Odd?


\medskip

\hspace{.2in}\textbf{a}
$(\begin{smallmatrix}
1 & 2 & 3 & 4 & 5 & 6 & 7 & 8 & \\
7 & 4 & 1 & 5 & 6 & 2 & 3 & 8 &
\end{smallmatrix})$

\medskip

We have to put this into disjoint cycles:

\begin{center}
$(\begin{smallmatrix}
1 & 2 & 3 & 4 & 5 & 6 & 7 & 8 & \\
7 & 4 & 1 & 5 & 6 & 2 & 3 & 8 &
\end{smallmatrix})$		$\mapsto(173)(2456)$.
\end{center}

\medskip

Now we can put the disjoint cycles to product of transpositions which will help us decide if it is even or odd:

\medskip
$(173)(2456)$		$\mapsto(13)(17)(26)(25)(24)$   $\longrightarrow$ this is Odd


\medskip
\bigskip


\hspace{.2in}\textbf{b} \hspace{.1in} $(71864)$		$\mapsto(74)(76)(78)(71)$	$\longrightarrow$  Even

\medskip

\hspace{.2in}\textbf{c} \hspace{.1in} $(12)(76)(345)$	 $\mapsto(12)(76)(35)(34)$ 		$\longrightarrow$ Even

\medskip

\hspace{.2in}\textbf{d} \hspace{.1in}  $(1276)(3241)(7812)$ 		$\mapsto (16)(17)(12)(31)(34)(32)(72)(71)(78)$		$\longrightarrow$ Odd

\medskip

\hspace{.2in}\textbf{e} \hspace{.1in} $(123)(2345)(1357)$  	$\mapsto (13)(12)(25)(24)(23)(17)(15)(13)$ $\longrightarrow$ Even

\medskip


%%%%%%%%%%%%%%%%%%%%%%%%%%%%%%%%%%%%%%%%%%%%%%%%%%%%
\newpage


\textbf{10a)} We learned in lecture that every permutations has a (unique) expression as a product os disjoint cycles. 
Transposition means $(a_1, a_2, a_3,......a_k)=(a_1a_k)(a_1a_k-1).....$. When we are given any permutations, we can turn those permutations into a product of transpositions.

\bigskip

\textbf{b)}

Assume that the set $T_1=\{(12),(13), . . . (1n)\} $ generates  $Sn.$

 Let a,b $\in$ \{2,3....,n\}
 then with the definition of transpositions: $$(ab)=(1a)(1b)(1a)....$$
 which it follows that (ab) $\in$  $<T_1>$.
 
 $<T_1>=S_n$, because we proved in the previous problem that transpositions in $S_n$ generates $S_n$.
 





					   
\end{document}