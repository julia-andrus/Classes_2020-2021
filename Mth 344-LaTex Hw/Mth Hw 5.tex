\documentclass{article}

\usepackage{amsmath}
\usepackage{xcolor}%text color
\usepackage{soul} %highlight text
\usepackage{amssymb} %gives us different symbols, %\therefore....etc
\usepackage{amsfonts}%gives you real number R Z Q...
\title{Mth Hw 5}
\date{}


\begin{document}

\maketitle

\textbf{1)} Assume $G_1$ and $G_2$ are groups, let $f: G_1 \mapsto  G_2 $ be an isomorphism, therefore it is a bijection and it respects  the operation. 

\medskip


Prove $f(e_1)=e_2$: 


\begin{center} 
$f(e_1)=f(e_1*e_1)$ 		$\mapsto$ (Def) \\

\medskip
$f(e_1)^{-1}f(e_1)=f(e_1)^{-1}f(e_1)*f(e_1)		\mapsto$ (Inverse Rule) \\

\medskip
$e_2=f(e_1)*e_2		\mapsto$ (Iden Rule) \\

\medskip
$e_2=f(e_1) 		 \surd$
\end{center} 

$$\therefore f(e_1)=e_2$$

\bigskip
\newpage
%%%%%%%%%%%%%%%%%%%%%%%%%%%%%%%%
\textbf{2)} Assume G is the symmetric group $S_3$ and H is the group of the matrices on pg. 28.
Prove G $\cong$ H. 

\medskip
First the matrices table on pg. 28 is :

\begin{table}[ht]
	\begin{tabular}{|c| |c| |c| |c| |c| |c| |c|}		
	\hline
	$ *$ & I & A & B & C & D & K \\
	\hline
	I & \hl{I} & A & B & C & D & K \\
	\hline
	A & A &\hl{I} & C & B & K & D \\
	\hline
	B& B & K  & D & A & \hl{I} & C \\
	\hline
	C& C & D  & K & \hl{I} & A & B \\
	\hline
 	D& D & C  & \hl{I} & K & B & A \\
	\hline
	K& K & B  & A & D & C & \hl{I}
	\end{tabular}
\end{table} \hspace{.01in} This table: has 6 elements,  has 4 self-inverses, and is is abelian.

\medskip


For $S_3$ table:

First:

Let: $\varepsilon= (123)$,
$\beta=(213)$,
$\delta=(132)$,
$\alpha=(321)$,
$\rho=(231)$,
$\gamma =(312)$
\begin{table} [ht]
\begin{tabular} {|c| |c| |c| |c| |c| |c| |c|}

\hline
$\circ$ & $\varepsilon$ & $\beta$ & $\delta$ & $\alpha$ & $\rho$  & $\gamma$ \\

\hline
$\varepsilon$ & \hl{$\varepsilon$} & $\beta$ &  $\delta$ & $\alpha$  & $\rho$ &  $\gamma$ \\

\hline
$\beta$ & $\beta$  & \hl{$\varepsilon$} & $\rho$ & $\gamma$ & $\delta$ & $\alpha$ \\

\hline
$\delta$ &$\delta$ &  $\gamma$ & \hl{$\varepsilon$} & $\rho$ &  $\alpha$ & $\beta$ \\

\hline
$\alpha$ & $\alpha$ & $\rho$ & $\gamma$ & \hl{$\varepsilon$} & $\beta$ & $\delta$ \\

\hline
$\rho$ & $\rho$ & $\alpha$ & $\beta$ & $\delta$ & $\gamma$ & \hl{$\varepsilon$}  \\ 

\hline
$\gamma$ & $\gamma$ & $\delta$ & $\alpha$ & $\beta$ & \hl{$\varepsilon$} & $\rho$
\end{tabular}
\end{table}


This table: has 6 elements,  has 4 self-inverses, and is \textbf{non}-abelian.
\medskip

$\therefore$ G is not isomorphic to  H
\newpage
%%%%%%%%%%%%%%%%%%%%%%%%%%%%%%

\textbf{3)} Prove which groups are isomorphic:

$S_3$ \hspace{.1in} $\mathbb{Z} _6$  \hspace{.1in} $\mathbb{Z}_3\times \mathbb{Z}_2$ \hspace{.1in} $\mathbb{Z}_7^{*} $
\medskip


Proof:
\medskip

For $\mathbb{Z}_6$, our table will look like this:


\begin{table}[ht]
	\begin{tabular}{|c| |c| |c| |c| |c| |c| |c|}		
	\hline
	$ +$ & 0 & 1 & 2 & 3 & 4 & 5 \\
	\hline
	0 & \hl{0} & 1 & 2 & 3 & 4 & 5 \\
	\hline
	1 & 1&  2& 3 & 4 & 5 & \hl{0} \\
	\hline
	2& 2 & 3  & 4 & 5& \hl{0} & 1 \\
	\hline
	3 & 3 & 4  & 5 & \hl{0} & 1 & 2 \\
	\hline
 	4 & 4 & 5  & \hl{0} & 1 & 2 & 3 \\
	\hline
	5& 5 & \hl{0}  & 1 & 2 & 3 & 4 
	\end{tabular}
\end{table}
This table: has 6 elements, has 2 self-inverses, and is abelian.

\medskip

For $S_3$: 3

As we stated in lecture, symmetric groups usually are non-abelian groups. 



Lets use the table from previous problem: 

\begin{table} [ht]
\begin{tabular} {|c| |c| |c| |c| |c| |c| |c|}

	\hline
	$\circ$ & $\varepsilon$ & $\beta$ & $\delta$ & $\alpha$ & $\rho$  & $\gamma$ \\

	\hline
	$\varepsilon$ & \hl{$\varepsilon$} & $\beta$ &  $\delta$ & $\alpha$  & $\rho$ &  $\gamma$ \\

	\hline
	$\beta$ & $\beta$  & \hl{$\varepsilon$} & $\rho$ & $\gamma$ & $\delta$ & $\alpha$ \\

	\hline
	$\delta$ &$\delta$ &  $\gamma$ & \hl{$\varepsilon$} & $\rho$ &  $\alpha$ & $\beta$ \\

	\hline
	$\alpha$ & $\alpha$ & $\rho$ & $\gamma$ & \hl{$\varepsilon$} & $\beta$ & $\delta$ \\

	\hline
	$\rho$ & $\rho$ & $\alpha$ & $\beta$ & $\delta$ & $\gamma$ & \hl{$\varepsilon$}  \\ 

	\hline
	$\gamma$ & $\gamma$ & $\delta$ & $\alpha$ & $\beta$ & \hl{$\varepsilon$} & $\rho$
	\end{tabular}
\end{table}

As you can see this table:

-has 6 elements

-has 4 self-inverses

-and is non-abelian
\medskip


Lets look at  $\mathbb{Z}_7^{*}$.  $\mathbb{Z}_7^{*}=\{0, 1, 2, 3, 4, 5, 6\}$. In the previous chapters we stated that operations of additions and multiplications are commutative which means they are abelian.

\begin{table} [ht]
\begin{tabular} {|c| |c| |c| |c| |c| |c| |c| |c|}
\hline
$*$ & 0 & 1 & 2 & 3 & 4 & 5 & 6 \\
\hline
0 & \hl{3} & 4 & 5 & 6 & 0 & 1 & 2 \\
\hline
1 & 4 & 5 & 6 & 0 & 1 & 2 & \hl{3} \\
\hline
2 & 5 & 6 & 0 & 1 & 2 & \hl{3} & 4  \\
\hline
3 & 6 & 0 & 1 & 2 & \hl{3} & 4 & 5 \\
\hline
4 & 0 & 1 & 2 & \hl{3} & 4 & 5 & 6 \\
\hline
5 & 1 & 2 & \hl{3} & 4 & 5 & 6 & 0 \\
\hline
6 & 2 & \hl{3} & 4 & 5 & 6 & 0 & 1
\end{tabular}
\end{table}

As you can $\mathbb{Z}_7^{*}$ generated by 3, it's cyclic! 

\medskip


For $\mathbb{Z}_3\times \mathbb{Z}_2$:
\medskip

 $\mathbb{Z}_3=\{0, 1, 2\}$ and $\mathbb{Z}_2=\{0, 1\}$
 
 \medskip
 then

\medskip

$\mathbb{Z}_3\times \mathbb{Z}_2=\{ (0,0), (1,0), (1,1), (2,0), (2,1), (0,1), (0,2) \}$.
When making this table you will find out that $\mathbb{Z}_3\times \mathbb{Z}_2$ is a cyclic order of 6, which is the same with $\mathbb{Z}_6$.

\medskip
After investigating each of these groups we can say that   $\mathbb{Z}_3\times \mathbb{Z}_2$,  $\mathbb{Z}_7^{*}$, and $\mathbb{Z}_6$ are isomorphic. 


\medskip
\newpage
%%%%%%%%%%%%%%%%%%%%%%%%%%%%%%%%%%%%

\textbf{4)}
Assuming E is a group of all even integers with respect to addition.

Prove $\mathbb{Z} \cong$E.
\medskip

For $\mathbb{Z} \cong$E it has to pass the following steps:

1) It has to be 1-1

2) It has to be onto

3) It has to respect the operation (addition)
\medskip


Proof:
\medskip

1) Let $f(x)=2x$. a,b $\in$ $\mathbb{Z}$. Assume f(a)=f(b)

then $$ f(a)=f(b)$$
$$2a=2b$$
$$a=b\surd$$

Yes it is 1-1!
\medskip


2) Fix any y$\in$ $\mathbb{Z}$.

\textbf{Scratch work:}
$$f(x)=y$$
$$2x=y$$
$$x=y/2$$

let $$x=y/2$$
then $$f(x)=2(y/2)$$
$$=y\surd$$

It is onto!

\medskip

3) With our definition: f(x+y)=f(x)+(y)
\medskip

then
$$f(x+y)=f(2x+2y)$$
$$=2x+2y$$
$$=f(x)+f(y)\surd$$

It does respect the operation (addition)!

\medskip


$\therefore$  $\mathbb{Z}$ $\cong$ E!

\newpage
%%%%%%%%%%%%%%%%%%%%%%%%%%%%%%%%
\textbf{5)}  Assume G be the groupf \{$10^{n} : n\in Z$\} with respect to multiplication.
Prove  G$\cong$ $\mathbb{Z}$.

\medskip
Proof:

\medskip
1) Is it 1-1? Let f: G $\mapsto$ $\mathbb{Z}$ and  $f(10^{n})=n$ and  $f(10^{m})=m$. Let x,y $\in$ G where $x=10^{n}$ and $y=10^{m}$ for some n,m $\in$ $\mathbb{Z}$


then
$$f(x)=f(y)$$
$$10^{n}=10^{m}$$
$$n=m$$

It is 1-1!
\medskip

2) Is it onto? Let y$\in$ Z, x$\in$ G

Let $x=10^{y}$

then $$f(x)=f(10^{y})$$
$$=y$$

It is onto!
\medskip

3) Respect the operation (multiplication)?
\medskip


With $x=10^{n}$ , $y=10^{m}$
\medskip
then 

$$f(xy)=f(10^{nm})   \mapsto (Theorem  1: Exponent Notation) $$
$$=f(10^{n+m})$$
$$=n+m$$
$$=f(10^{n})+f(10^{m}$$

Yes it does!
\medskip


$ \therefore$ G $\cong$ $\mathbb{Z}$ ! 


%%%%%%%%%%%%%%%%%%%%%%%%%%%%%
\newpage

\medskip
\textbf{6)} Assume G is any group and a $\in$ G. Prove $f(x)=axa^{-1}$ is an automorphism of G.  

\medskip 

To prove that f(x) is an automorphism of G:


1) it's 1-1


2) it's onto


3) $f(a*_1b)=f(a)*_2f(b)$
\medskip

Proof: 
Let $a^{-1}(a(f(x)))=f_{a^{_-1}}xa^{-1})$
\medskip


then


$$f(x)=axa^{-1}$$
$$a^{-1}(a(f(x)))=f_{a^{_-1}}(axa^{-1})$$
$$=a^{-1}xa^{-1}a$$
$$=(a^{-1}a)x(a^{-1})(a)$$
$$=(e)x(e)$$
$$=x$$

$f_a(f_{a^{-1}}(x))$ will also end up with x:
$$f_a(f_{a^{_-1}}(x))$$
$$f_a(a^{-1}xa)$$
$$=aa^{-1}xaa^{-1}$$
$$(aa^{-1})x(aa^{-1})$$
$$=(e)x(e)$$
$$=x$$

This method was to show that $f_a$ is 1-1 and onto.

\medskip


To show that it respects the operation:

\medskip

Fix any a$\in$ G and let x,y $\in$ G

then $$f(xy)=axya^{-1}$$
$$=axeya^{-1} $$
$$=ax(a^{-1}a)ya^{-1}$$
$$=(axa^{-1})(aya^{-1})$$
$$=(ex)(ey)$$
$$=f(x)f(y)\surd$$
\medskip
$\therefore$ $f(x)=axa^{-1}$ is an automorphism of G.


%%%%%%%%%%%%%%%%%%%%%%%%%%%%%%%%%%%
\newpage

\textbf{7)}  We have $S_6$. What is the order?

$$(\begin{smallmatrix}
1 &2 &3 &4 &5 &6 \\
6 & 1 &3 &2 &5 &4 
\end{smallmatrix})$$

Make this into disjoint cycles:

$$(1642)(3)(5) $$

With this, our least common multiple: (4, 1, 1) =4.

\medskip

So that means that 

\medskip
\begin{center}
$(\begin{smallmatrix}					 
1 &2 &3 &4 &5 &6 \\
6 & 1 &3 &2 &5 &4 
\end{smallmatrix})^{4}$     $\mapsto$ $(\begin{smallmatrix}
					1 &2 &3 &4 &5 &6 \\
					1 &2 &3 &4 &5 &6 
					\end{smallmatrix})$
\end{center}
					
					\medskip
					
$\therefore$ Order of $S_6$ is 4.
%%%%%%%%%%%%%%%%%%%%%%%%%%%%%%%%%%%%

\newpage

\textbf{8)} 	Assume A is the set of all real numbers $x \neq 0, 1, 2$.  
Find the the order in $S_A$ of 
$$f(x)=2/2-x$$


f isn't defined at 0, 1, 2.  We have to find the smallest positive integer. 



We can try with x being a number:
\medskip

x=3

$$f(x)=2/2-x$$
$$=2/2-3$$
$$=-2$$

let x=4

$$f(x)=2/2-4$$
$$=-1$$

We can also try it with rational numbers:

let x=1/2

$$f(x)=2/2-1/2$$
$$=.333$$



%%%%%%%%%%%%%%%%%%%%%%%%%%%%%%%%%%%
\newpage
\textbf{9)} Explanation: Lets think of $Z_n=\{0, 1, ...,n-1\}$:

\bigskip

Suppose  $<Z_2,+>$:
\medskip
\bigskip


Let $Z_2$=\{0,1\}, under addition mod 2. Let G be the additive group of $x^{n}$ from Z2.
 G is infinite because n is every number and $x^{n}$ is a distinct element G. Every element of G has order 2.
%%%%%%%%%%%%%%%%%%%%%%%%%%%%%%%%%%
\newpage

\textbf{10)}  We know that $\mathbb{Z}_{24}=\{0,1,2.....24\}$. From the definition of Order, we have n=24 and we want to list elements m is the smallest positive number $(1<m<24)$.

\medskip

a) order 2
\medskip

(m,n)=(m,24)=24/2=12.
\medskip

Since 12 has order 2 in $\mathbb{Z}_{24} $
\medskip

then $$<12>=\{0,12\}$$
\medskip


b) order of 3
\medskip

(m,n)=(m,24)=24/3=8
\medskip

Since 8 has order 3 in $\mathbb{Z}_{24}$
\medskip

then $$<8>=\{0,8,16\}$$

\medskip
c) order 4
\medskip

(m,n)=(m,24)=24/4=6

\medskip

Since 6 has order 4 in $\mathbb{Z}_{24}$
\medskip

then $$<6>=\{0, 6, 12, 18\}$$


\medskip
d) order of 6


(m,n)=(m,24)=24/6=4
\medskip

Since 4 has order 6 in $\mathbb{Z}_{24}$

\medskip
then $$<4>=\{0, 4, 8, 12, 16, 20\}$$


%%%%%%%%%%%%%%%%%%%%%%%%%%%%%%%%%%%%

\newpage

\textbf{11)} Proof:  From our definition of Order in the class videos, let $ord(a)=n$

then $a^{n}=e$
so $$(a^{k})^{n}=a^{kn}$$
$$=(a^{n})^{k}$$
$$=e^{k}$$
$$=e$$

Assume that $a^{k}=m$
then $a^{km}=e$ such that m is the smallest number. 

If m does divide n 
the $n=mq+r$ where $r<m$



then $$e=a^{nk}=(a^{k})^{mq+r}$$
$$=a^{kmq+r}$$
$$=a^{kmq}(a^{r}$$
$$=ea^{r}$$
$$=a^r$$
Our assumption is false because m is no the smallest number ($r<m$) such that $a^{m}=e$.

\medskip

$\therefore$ order of $a^{k}$ is a divisor of the order of a.

%%%%%%%%%%%%%%%%%%%%%%%%%%%%%%%%%%%%%%%%%%
\newpage

\textbf{12)} Proof: Definition: $a^{m}=e$ for some m $\in N$, then ord(a)= the least n $\in N$ such that $a^{n}=e$.

\medskip

Since ord($a^{k}$)=m
then $$(a^{k})^{m}=e$$
$$a^{km}=e$$

 Let ord($a^{k}$)=x
 
 \medskip
 
 then $$a^{kx}=e$$
 
 \medskip
 
with the divisor rule: km/kx		$\mapsto$ m/x
 
 
 \medskip
 $\therefore$ ord($a^{k}$)=m


























 
 \end{document}