\documentclass{article}

\usepackage{amsmath}
\usepackage{soul}
\usepackage{amssymb}
\usepackage{xcolor}
\usepackage[T1]{fontenc}
\usepackage[ttdefault=true]{AnonymousPro}
\title{Mth Hw 6 Redo} 
\date{}


\begin{document}


\maketitle


\textbf{1)} Suppose $\mathbb{Z}_{16}$=\{1, 2, 3,...15\}. List elements of $<6>$.

\medskip
Since we are listing elements of $<6>$
then  

$$6=6$$
$$6+6=12$$
$$6+6+6=18=2$$
$$\hspace{.25in} =8$$
$$\hspace{.65in} =30=14$$
$$\hspace{.9in} =36=20=4$$
$$\hspace{.97in} =42=26=10$$
$$\hspace{.9in} =48=32=0$$



So 
$$<6>=\{0, 2, 4, 6, 8, 10, 12, 14\}$$

%%%%%%%%%%%%%%%%%%%%%%%%%%%%%%%%%%%%%%%%%%%%%%%%%%%%

\newpage

\textbf{4)} Assume in any cyclic group of order n, there are elements of order k for every integer k which divides n. 

\medskip

Proof: By the cyclic definition given in class:suppose G is a cyclic group generated by a (cyclic def). With our Order definition, we have $G=a^{n}=e$,. Since G is cyclic, it must be generated by a single element.

\medskip

Let n=kx for some integer x. Then $g^{x}$ is an element of order k. 

\medskip

$$ord(g^{x})=n/gcd(n,x)=kx/gcd(kx,x)=kx/x=k$$
\medskip
%%%%%%%%%%%%%%%%%%%%%%%%%%%%%%%%%%%%%%%%%%%%%%%%%%%%

\newpage

\textbf{6)} Example: Suppose we have $\mathbb{Z}_{2}X\mathbb{Z}_{2}$

\medskip

We know that $\mathbb{Z}_{2}$=\{0,1\}

then  $$\mathbb{Z}_{2}X\mathbb{Z}_{2}=\{(0,0),(1,0),(0,1),(1,1)\}$$

\begin{table}[ht]
	\begin{tabular}{|c| |c| |c| |c| |c|}	
	\hline
	& (0,0) & (1,0) &  (0,1)  &(1,1) \\
	\hline
	(0,0) & (0,0) & (0,1) &  (1,0)  &(1,1) \\
	\hline
	(1,0) & (1,0) &(0,0) &(1,1) & (0,1) \\
	\hline
	(0,1) & (0,1) & (1,1) & (0,0) & (1,0) \\
	\hline
	(1,1) & (1,1) & (0,1) & (1,0) & (0,0) 
	\end{tabular}
\end{table}

Orders: $$ (0,0)=1$$
$$(1,0)=2$$
$$(0,1)=2$$
$$(1,1)=2$$

$\mathbb{Z}_{2}$ is cyclic, but $\mathbb{Z}_{2}X\mathbb{Z}_{2}$ is not a cyclic group because every element of  $\mathbb{Z}_{2}X\mathbb{Z}_{2}$ has order 1 or 2.
%%%%%%%%%%%%%%%%%%%%%%%%%%%%%%%%%%%%%%%%%%%%%%%%%%%%


\newpage
\textbf{7)} $A_{n}=\{x \in \mathbb{Q} : n <x < n+1\}$. Our definition of partition is a set of any decomposition of A into subset that is 

1) nonempty

2) disjoint

3) cover all A

\medskip

-Nonempty: For any n $\in \mathbb{Z}$, there is n $\in A_{n}, so  A_{n}$ is nonempty.
\medskip


-Disjoint: Let m,n $\in \mathbb{Z}$ with m $\neq$ n. Let  $m < n$. If $A_{m} \bigcap A_{n}= \phi$, then there exists
some rational number x in $ A_{m} \bigcap A_{n}$. But then $m \leq x < m + 1$ and $n \leq x < n + 1$. Since
$m + 1 \leq n$, this implies $x < x$, which is impossible.

\medskip

-Cover all A: Fix any r in Q. Let $n = brc$, the floor of r. Then $n \leq r < n + 1$ and n in Z so r in An.
Therefore $Q \subseteq \bigcup_{n} \in ZA_{n}$ and the family $\{An : n \in Z\}$ covers Q.


\medskip


Equivalence relation: For any $r, s \in Q, r \sim$ s iff brc = bsc.


%%%%%%%%%%%%%%%%%%%%%%%%%%%%%%%%%%%%%%%%%%%%%%%%%%%%

\newpage

\textbf{9)} f $\sim$ g iff f(0)=g(0). Prove $\sim$ is an equivalence relation and describe the partition.

1) Reflex? Let x $\in$ $\mathbb{R}$ 
then f(0)=f(0)

so $$f \sim f \surd$$.

\medskip

2) Symm? Let f,g $\in$ F($\mathbb{R})$. Assume f $\sim$ g, then f(0)=g(0).

So then g(0)=f(0). 

\medskip

3) transitivity? Fix f,g,h $\in$ F($\mathbb{R})$. Assume f $\sim$ g, g$\sim$ h

then f(0)=g(0) and g(0)=h(0), so f(0)=h(0)



Therefore $$x  \sim z\surd$$

\begin{center}
$\therefore \sim$ is reflex., symm., and has transivity 
\end{center}

For the partition:The equivalence relation groups the functions into infinitely many classes, according to the y-intercept. 
%%%%%%%%%%%%%%%%%%%%%%%%%%%%%%%%%%%%%%%%%%%%%%%%%%%%


\newpage

\textbf{10)}  We have $A_{r}$=\{$(x,y) : y=2x+r$\}, where r $\in \mathbb{R}$ ($\mathbb{R}$=\{-1/2,0,1/2,1,$\pi$...\}).
 Prove $\mathbb{R}X\mathbb{R}$ is a partition. 

\medskip

-Nonempty: Let r $\in \mathbb{R}$, we have (0,r)$\in A_{r}$, therefore $A_{r}$ is nonempty

\medskip


-Disjoint: Let r, s  in $\mathbb{R}$ ($r \neq s$). If $A_{r} \bigcap A_{s} \neq \phi$ then there exists some ordered pair
$(x, y) \in A_{r}  (intersection)  A_{s}$. But  r equal y minus 2x equal s, which is impossible.

\medskip

-Cover allA:  Fix any ordered pair $(x, y) \in \mathbb{R\text{x}R}$. Let r equal y minus 2x. Then $y = 2x+r and (x, y) \in A_{r}$.
Therefore$\mathbb{R\text{x}R} \subseteq \bigcup \in _\mathbb{R}A_{r}$ and the family $\{Ar : r \in R\}$ covers $\mathbb{R\text{x}R}$.

\medskip

Equivalence relation: For any (x, y),(a, b)$ \in \mathbb{R\text{x}R}$,
$(x, y) \sim (a, b)$ iff y minus 2x equals  b minus 2a.

%%%%%%%%%%%%%%%%%%%%%%%%%%%%%%%%%%%%%%%%%%%%%%%%%%%%
\newpage

\textbf{11)} Proof:

\medskip

1) Reflex? Since G is a group, there's an identity with e $\in$ G. Assume a $\sim$ a

\medskip

then $$a=xax^{-1}$$
$$a=eae^{-1}$$
$$a=a \surd$$

2) Symm? Assume a $\sim$ b and b $\sim$ a. Let a,b $\in G$. There exists x $\in G$ such that $a=xbx^{-1}$. But $b=yay^{1}$ where $y=x^{-1}$, so $b\sim a$


\medskip
\medskip


3) transitivity? let a,b,c $\in$ G. Assume a $\sim$ b and b $\sim$ c.Then there exists x,y $\in G$ such that the $a=xax^{-1}$ and $b=ycy^{1}$By substitution, we have $(xy)c(xy)^{-1}$ so a $\sim$ c

\medskip

$\therefore \sim$ is reflex., symm., and has transivity

\medskip

\medskip

Now describing equivalence class of e: 

\medskip

We have e $\in$ G, $[e]=\{y\in G| y \sim e\}$


\hspace{1.17in}= \{y $\in$ G | there exists x $\in$ G with y equals $(xex)^{-1}= e$\}


\hspace{1.17in}=$\{e\}$

%%%%%%%%%%%%%%%%%%%%%%%%%%%%%%%%%%%%%%%%%%%%%%%%%%%%
\newpage

\textbf{12)} Assume G is a group. In G, let a $\sim$ b iff there is a nonzero
integer k such that $a^{k} = b^{k}$. 
Prove that $\sim$ is an equivalence relation and describe the equivalence class of e. Since G is a group then e $\in$ G, 

\medskip

1) reflex? Fix any a $\in$ G. Assume a $\sim$ a


then  $a^{k} = a^{k}$

try $a^{1}=a^{1}$

it works, $a \sim a$

\medskip

2) symm? Fix any a,b $\in$ G. Assume a $\sim$ b. There exists nonzero $k \in \mathbb{Z}$ such that $a^{k}=b^{k}$. Since $a^{k}=b^{k}$ is the same thing as saying $b^{k}=a^{k}$, so a $\sim b$


If a $\sim$ b, then b $\sim$ a

Because of the equal sign b $\sim$ a has to be true. 


\medskip


3) transitivity? Fix a,b,c $\in$ G. Assume a $\sim$ b and b $\sim$ c. There exists j, k $\in \mathbb{Z}$ such that $a^{j}=b^{j}$ and $b^{k}=c^{k}$. Since j and i are non-zeros, so is jk. So a $\sim$ c.

$$a^{jk}=(a^{j})^{k}=(b^{j}0^{k}=(b^{k})^{j}=(c^{k})^{j}=c^{jk}$$


\medskip

$\therefore \sim$ is reflex., symm., and has transivity

\medskip

Now describing equivalence class of e: 

\medskip


$[e]=\{y\in G | y\sim e\}$


\hspace{.17in}=\{$y \in G$ | there exists k $\neq 0$ such that $y^{k}=e^{k}$\}


\hspace{.17in}=\{$y \in G$ | there exists k $\neq 0$ such that $y^{k}=e$\}


\hspace{.17in}=\{$y in G$ | ord(y) < infinity\}





   
 
%%%%%%%%%%%%%%%%%%%%%%%%%%%%%%%%%%%%%%%%%%%%%%%%%%%%


\end{document}