\documentclass{article}

\usepackage{amsmath}
\usepackage{soul}
\usepackage{amssymb}
\usepackage{xcolor}

\title{Mth Hw 6} 
\date{}

\begin{document}

\maketitle 

\textbf{1)} Suppose $\mathbb{Z}_{16}$=\{1, 2, 3,...15\}. List elements of $<6>$.

\medskip
Since we are listing elements of $<6>$
then  

$$6=6$$
$$6+6=12$$


So 
$$<6>=\{0, 6, 12\}$$
%%%%%%%%%%%%%%%%%%%%%%%%%%%%%%%%%%%%%%%%%%%%%%%%%%%%
\newpage

\textbf{2)} List elements of $<f>$ in $S_6$.

\medskip
We are given:

$$(\begin{smallmatrix}
1&2&3&4&5&6 \\
6&1&3&2&5&4
\end{smallmatrix})$$

We can put this into disjoint cycles:
$$(1642)(3)(5)$$
$$(1642)$$

Now to get the list of elements:
$$(1642)^{1}=(1642)$$
$$(1642)^{2}= (14)(26)$$
$$(1642)^{3}=(1246)$$
$$(1642)^{4}=(1)$$

So

$$<f>=\{(14)(62), (1246), (1)\}$$
%%%%%%%%%%%%%%%%%%%%%%%%%%%%%%%%%%%%%%%%%%%%%%%%%%%%

\newpage

\textbf{3)} Prove that every cyclic group is abelian.

\medskip

Proof:


\medskip

Note: We stated earlier in lecture that abelian is same thing as a group being commutative. So with this idea, suppose G is cyclic and g is the generator in G $(G=<g>).$


\medskip

Fix a,b $\in$ G
then

$$a=g^{n}$$
$$b=g^{m}$$
then
\begin{center}
\hspace{1.3in}$ab=ba$ \hspace{.1in} (commu def: ab=ba)
\end{center} 
$$g^{n}g^{m}=g^{m}g^{n}$$
\begin{center}
\hspace{1.3in} $g^{n+m}=g^{m+n}$  \hspace{.1in} (exponential rule)
\end{center}

So $$ab=ba \surd$$

\newpage
%%%%%%%%%%%%%%%%%%%%%%%%%%%%%%%%%%%%%%%%%%%%%%%%%%%%

\textbf{4)} Assume in any cyclic group of order n, there are elements of order k for every integer k which divides n. 

\medskip

Proof: By the cyclic definition given in class:suppose G is a cyclic group generated by a (cyclic def). With our Order definition, we have $G=a^{n}=e$,. Since G is cyclic, it must be generated by a single element.

\medskip
We can have order(a)=n and order(b)=k  (order def), so let $ a,b \in G$ then $$G=a^{n}=e$$ and $$G=b^{k}=e$$


Now k divides n, then using the Division Algorithm: k divided n everywhere, where k $\in \mathbb{Z}$.

 
$$\hspace{.1in}n=kq+r$$
$$a^{n}=a^{kq+}$$
\begin{center}
\hspace{.9in} $e=a^{kq}a^{r}$  \hspace{.1in} (0 $\leq$ r $<$n)%%center at hspac{.9} to math $$
\end{center}
$$e=ea^{r}$$
$$e=a^{r}$$
Since (0 $\leq$ r $<$n), then r=0 which means that $k|n$

\medskip

$$\therefore k|n $$

\medskip
%%%%%%%%%%%%%%%%%%%%%%%%%%%%%%%%%%%%%%%%%%%%%%%%%%%%
\newpage

\textbf{5)} Assume $GXH$ is a cyclic group, then G and H are both cyclic.


Recall that G is cyclic iff $G=<a>$ for some a $\in$ G, we call generator for G
\medskip

Proof: We have $GXH$. $GXH=<a,b>$, where $G=<a>$ and $H=<b>$ 
  then  every (x,y) $\in$ %$\mathbb{Z}$ and x $\in$ G, y $\in$ H
   $GXH$ where it can be written as: 
   
   \begin{center}
   $\hspace{1in}(x,y)^{k}=(a,b)^{k}$  \hspace{.1in} for some $(k \in \mathbb{Z})$
   \end{center}
   so $$(x,y)=(a^{k},b^{k}) $$
   then we have $$x=a^{k} , y=b^{k}$$
   then $$G=<a>$$ and $$H=<b>$$ 
   
   This passes our definition where $G=<a>$ and $H=<b>$.
%%%%%%%%%%%%%%%%%%%%%%%%%%%%%%%%%%%%%%%%%%%%%%%%%%%%
\newpage

\textbf{6)} Example: Suppose we have $\mathbb{Z}_{2}X\mathbb{Z}_{2}$

\medskip

We know that $\mathbb{Z}_{2}$=\{0,1\}

then  $$\mathbb{Z}_{2}X\mathbb{Z}_{2}=\{(0,0),(1,0),(0,1),(1,1)\}$$

\begin{table}[ht]
	\begin{tabular}{|c| |c| |c| |c| |c|}	
	\hline
	& (0,0) & (1,0) &  (0,1)  &(1,1) \\
	\hline
	(0,0) & (0,0) & (0,1) &  (1,0)  &(1,1) \\
	\hline
	(1,0) & (1,0) &(0,0) &(1,1) & (0,1) \\
	\hline
	(0,1) & (0,1) & (1,1) & (0,0) & (1,0) \\
	\hline
	(1,1) & (1,1) & (0,1) & (1,0) & (0,0) 
	\end{tabular}
\end{table}

Orders: $$ (0,0)=1$$
$$(1,0)=2$$
$$(0,1)=2$$
$$(1,1)=2$$

$\mathbb{Z}_{2}X\mathbb{Z}_{2}$ is not cyclic because  $\mathbb{Z}_{2}X\mathbb{Z}_{2}$ is supposed to a cyclic order of 4, but it has a cyclic order of 2. 


%%%%%%%%%%%%%%%%%%%%%%%%%%%%%%%%%%%%%%%%%%%%%%%%%%%%
\newpage
\textbf{7)} $A_{n}=\{x \in \mathbb{Q} : n <x < n+1\}$. Our definition of partition is a set of any decomposition of A into subset that is 

1) nonempty

2) disjoint

3) cover all A

\medskip

Let's see that this is saying: we have x $\in \mathbb{Q}$, where $\mathbb{Q}$=\{-1/2,-1/4,1/2....\}. And $A_{n}=\{x \in \mathbb{Q} : n <x < n+1\}$, where n $\in \mathbb{Z}$.

\medskip
We can try with examples:

\medskip

[-3]=\{-3+1,-3+4,-3+7...\}

\medskip

[1]=\{1+1,1+2,1+3.......\}

\medskip
 
As you can see we have made $[n]=\{x \in \mathbb{Q} : n <x < n+1\}$. Since $x \in \mathbb{Q}$ and it's between n integer and n+1; for any n $\in \mathbb{Z}$, it's going to be either bigger or smaller than any $x \in \mathbb{Q}$. This is a partition because each set is nonempty, disjoint and cover all A.
   
 
%%%%%%%%%%%%%%%%%%%%%%%%%%%%%%%%%%%%%%%%%%%%%%%%%%%%
\newpage
\textbf{8} Assume m $\sim$ n iff $m-n$ is a multiple of 10 (0, 10, 20...). Prove $\sim$ is an equivalence relation and describe the partition.



In class, we stated that for $\sim$ to be an equivalence relation it has to pass 3 steps:

\medskip

1) It have to be reflexive: x $\sim$ x


\medskip

2) it has to be symmetric: y $\sim$ x whenever x $\sim$  y



\medskip

3) it has transitivity: with x $\sim$  y and y $\sim$ x, then x $\sim$ z.

\medskip

Proof:
\medskip

We have $ <\mathbb{Z}, +>$. Let H=\{10n : n $\in$ $\mathbb{Z}$ \}

\medskip


1) Is it reflexive?  We need x $\sim$ x for any x $\in \mathbb{Z}$

then $x-x=0$ $\surd$

It is reflexive, because 0 is a multiple of 10.


\bigskip

2) Is it symmetric? If for all  x, y $\in \mathbb{Z}$, y $\sim$ x, whenever $x \sim y$.
\medskip


Proof: for $x \sim y \in H$

\medskip

then for some n $\in \mathbb{Z}$  $$x-y=10(n)$$

For y $\sim x \in H$

then $$y-x=10(-n)$$



Since -n is in $\mathbb{Z}$, then y $\sim$ x $\in$ H.
\medskip

So $$y \sim x \surd$$

\medskip

3) Fix x,y,z $\in \mathbb{Z}$.  Assume  x $\sim$ y=10n and y $\sim$ z=10n

\medskip

then   $x \sim y=10n_{1}$ and $y \sim z=10n_{2}$

So x$\sim$ z=(x-y)+(y-z)
$$=10n_{1}+10n_{2}$$
$$=10(n_{1}+n_{2}). (n_{1}+n_{2}\mathbb{Z})$$



So x$\sim$ z=10n


$$x\sim z \surd$$

$\therefore \sim$ is reflex., symm., and has transitivity
\newpage

Partition: It is true that there exists a partition when $\sim$ is an equivalence relation.

\medskip


(0,10,20,30....)


(1,11,21,31....)


(2,12,22,32....)


(3,13,23,33....)

.

.

.

.


.

.

(9,19,29,39....)

There is 10 possible sets for this partition


%%%%%%%%%%%%%%%%%%%%%%%%%%%%%%%%%%%%%%%%%%%%%%%%%%%%

\newpage

\textbf{9)} f $\sim$ g iff f(0)=g(0). Prove $\sim$ is an equivalence relation and describe the partition.

1) Reflex? Let x $\in$ $\mathbb{R}$ 
then f(x)=f(x)

so $$x \sim x \surd$$.

\medskip

2) Symm? Let x,y $\in$ $\mathbb{R}$. Assume x $\in$ y, then f(x)=g(y).

Now assume y $\in$ x , then f(y)=g(x).

If x $\in$ y, then y $\in$ x 

\medskip

3) transitivity? Fix x,y,z, $\in$ $\mathbb{R}$. Assume x $\sim$ y, y $\sim$ z

then x $\in \mathbb{ Z}$, because of the equal sign, so f(x)=g(z)

$$x  \sim z\surd$$
\begin{center}
$\therefore \sim$ is reflex., symm., and has transivity 
\end{center}

For the partition: We have f(x) and g(x) where f(x)=g(x).
So for example, in f(x)=\{-1/2,1,2,$\pi$...\} would match to the set of g(x).


%%%%%%%%%%%%%%%%%%%%%%%%%%%%%%%%%%%%%%%%%%%%%%%%%%%%


\newpage

\textbf{10)}  We have $A_{r}$=\{$(x,y) : y=2x+r$\}, where r $\in \mathbb{R}$ ($\mathbb{R}$=\{-1/2,0,1/2,1,$\pi$...\}).
 Prove $\mathbb{R}X\mathbb{R}$ is a partition. 

\medskip

Let's try out some examples: pick any integer x,y 


$$(x,y)=(3,5): y=2x+r$$
$$3=2(5)+r$$
$$-7=r$$


$$(x,y)=(-1/2,1/2)$$
$$3/2=r$$

$$(x,y)=(\pi,\pi/2)$$
$$-3\pi/2=r$$

\medskip

This works because r $\in \mathbb{R}$ and all those 3 choices results into the real numbers.  

Now the equivalence class would be that there would be a set of $\mathbb{R}$, where there are groups of subset that satisfy \{$(x,y) : y=2x+r$\}. For example, the 3 each groups of subsets of the examples, we did above, would satisfy the set. 


%%%%%%%%%%%%%%%%%%%%%%%%%%%%%%%%%%%%%%%%%%%%%%%%%%%%
\newpage

\textbf{11)} Proof:

\medskip

1) Reflex? Since G is a group, there's an identity with e $\in$ G. Assume a $\sim$ a

\medskip

then $$a=xax^{-1}$$
$$a=eae^{-1}$$
$$a=a \surd$$

2) Symm? Assume a $\sim$ b and b $\sim$ a

then $$a=xbx^{-1}$$ ,  $$b=xax^{-1}$$
since e $\in$ G 

then
$$a=xbx^{-1}$$
$$a=ebe^{-1}$$
$$a=b$$

Same idea for $$b=xax^{-1}$$
$$b=xax^{-1}$$
$$b=eae^{-1}$$
$$b=a$$

So a $$\sim b , b \sim \surd$$


3) transitivity? let a,b,c $\in$ G. Assume a $\sim$ b and b $\sim$ c

then 

$$a=xbx^{-1} ,  b=xcx^{-1}$$


Because of the identity e, we have a=b and b=c
if a=b and b=c
the a=c.



$\therefore \sim$ is reflex., symm., and has transivity

\medskip

Now describing equivalence class of e: 

\medskip

We have e $\in$ G, $[e]_{\sim}$=\{x $\in$ G : $a=xbx^{-1}$ \}

%%%%%%%%%%%%%%%%%%%%%%%%%%%%%%%%%%%%%%%%%%%%%%%%%%%%
\newpage

\textbf{12)} Assume G is a group. In G, let a $\sim$ b iff there is a nonzero
integer k such that $a^{k} = b^{k}$. 
Prove that $\sim$ is an equivalence relation and describe the equivalence class of e. Since G is a group then e $\in$ G, 

\medskip

1) reflex? Fix any a $\in$ G. Assume a $\sim$ a


then  $a^{k} = a^{k}$

\medskip

2) symm? Fix any a,b $\in$ G. Assume a $\sim$ b, then $$a^{k}=b^{k}$$.


If a $\sim$ b, then b $\sim$ a



Because of the equal sign b $\sim$ a has to be true. 


\medskip


3) transitivity? Fix a,b,c $\in$ G. Assume a $\sim$ b and b $\sim$ c, then a $\sim$ c because of the equal sign.

\medskip

$\therefore \sim$ is reflex., symm., and has transivity

\medskip

Now describing equivalence class of e: 

\medskip


Since k $\in$ $\mathbb{Z}$ and e $\in$ G, $[e]_{\sim}$=\{a,b $\in$ G : $a^{k}=b^{k}$\}




\end{document}