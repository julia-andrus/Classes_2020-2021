\documentclass{article}

\usepackage{amsmath}
\usepackage{xcolor}%text color
\usepackage{soul} %highlight text
\usepackage{amssymb} %gives us different symbols, %\therefore....etc
\usepackage{amsfonts}%gives you real number R Z Q...

\title{Hw 5 Redo}
\date{}

\begin{document}


\maketitle

Redo: 8, 9, 10

\medskip



\textbf{8)} 	Assume A is the set of all real numbers $x \neq 0, 1, 2$.  
Find the the order in $S_A$ of 
$$f(x)=2/2-x$$

Definition of order is ord(a)=n where $a^{n}=e$
f isn't defined at 0, 1, 2.  We have to find the smallest positive integer. 

\medskip

$f=2/2-x$

$f^{2}=2-x/1-x$

\medskip

$f^{3}=2-2/2$

\medskip

$f^{4}=$the identity 

\medskip

ord(f)=4 because that's how many times that it will get us to the begining. 


%%%%%%%%%%%%%%%%%%%%%%%%%%%%%%%%%%%%%%%%%%%%%%%%

\newpage

\newpage
\textbf{9)} Yes. Think of ord(0)=1 $\in \mathbb{Z}$ and ord(-1)=2 $\in \mathbb{R^{*}}$. $\mathbb{R^{*}}$ gives us infinite groups of real numbers, but only gives us a finite order. $\mathbb{Z}$ gives us finite groups of integers, but also will give us a finite order. 
 

 
 
 %%%%%%%%%%%%%%%%%%%%%%%%%%%%%%%%%%%%%%%%%%%%%%%%

\newpage
 
 
 \textbf{10)}  We know that $\mathbb{Z}_{24}=\{0,1,2.....24\}$. From the definition of Order, we have n=24 and we want to list elements m is the smallest positive number $(1<m<24)$.

\medskip

a) order 2
\medskip

(m,n)=(m,24)=24/2=12.
\medskip

So 12 has order 2 in $\mathbb{Z}_{24} $
\medskip

then $$<12>=\{12\}$$
\medskip


b) order of 3
\medskip

(m,n)=(m,24)=24/3=8
\medskip

Since 8 has order 3 in $\mathbb{Z}_{24}$
\medskip

then $$<8>=\{8,16\}$$

So 8,16 has order 3 in $\mathbb{Z}_{24} $


\medskip
c) order 4
\medskip

(m,n)=(m,24)=24/4=6

\medskip

Since 6 has order 4 in $\mathbb{Z}_{24}$
\medskip

then $$<6>=\{ 6, 12, 18\}$$



\medskip
d) order of 6


(m,n)=(m,24)=24/6=4
\medskip

Since 4 has order 6 in $\mathbb{Z}_{24}$.

\medskip
then $$<4>=\{4, 20\}$$

because 8, 12, and 16 had been pointed out in the previous problems. 


\end{document}