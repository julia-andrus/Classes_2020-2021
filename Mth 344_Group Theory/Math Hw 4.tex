\documentclass{article}

\usepackage{amssymb, amsmath, amsfonts}
\usepackage{color, soul}
%\author{J}
\title{Math Hw 4}
\date{\today}

\begin{document}
\maketitle

\textbf{1)} 
We know that:





\medskip


$f=(\begin{smallmatrix} 
1 & 2 & 3 & 4 & 5 & 6 \\ 
6 & 1 & 3 & 5 & 4 &2
\end{smallmatrix}) $
				$ \mapsto f^-1=(\begin{smallmatrix} 
				1 & 2 & 3 & 4 & 5 & 6 \\ 
				6 & 1 & 3 & 5 & 4 &2
				\end{smallmatrix}) $ 			

\bigskip
 

%\smallskip

$ g=(\begin{smallmatrix} 
1 & 2 & 3 & 4 & 5 & 6 \\									
2 & 3 & 1 & 6 & 5 & 4
\end{smallmatrix}) $
				$ \mapsto g^-1=(\begin{smallmatrix} 
				1 & 2 & 3 & 4 & 5 & 6 \\									
				3 & 1 & 2 & 6 & 5 & 4
				\end{smallmatrix}) $
					
	
\bigskip

	



$h=(\begin{smallmatrix} 
1 & 2 & 3 & 4 & 5 & 6 \\
3 & 1 & 6 & 4 & 5 & 2
\end{smallmatrix}\ )$
					$\mapsto h^-1=(\begin{smallmatrix}
					1 & 2 & 3 & 4 & 5 & 6 \\
					2 & 6 & 1 & 4 & 5 & 3
					\end{smallmatrix})$	

\bigskip

\hspace{.2in}\textbf{a)}

$g  \circ  f=(\begin{smallmatrix}
1 & 2 & 3 & 4 & 5 & 6 \\	
4 & 2 & 1 & 5 & 6 & 3
\end{smallmatrix})$

\medskip
				Check: \begin{center}
				$g(f(1))=g(6)=4 $ \\
				
				$g(f(2))=g(1))=2 $ \\
					 $\cdot$ \\
					$\cdot$ \\
					 $\cdot $\\
				$g(f(6))=g(2)=3 $					
				
				\end{center}

\bigskip

\hspace{.2in}\textbf{b)}

$h \circ g^-1 \circ f^-1=(\begin{smallmatrix}
1 & 2 & 3 & 4 & 5 & 6 \\	
4 & 6 & 1 & 5 & 2 & 3
\end{smallmatrix})$

\medskip
				Check: \begin{center}
				$h(g^-1(f^-1(1)))=h(g^-1(6)) =h(4)=4 $ \\
				
				$h(g^-1(f^-1(2)))=h(g^-1(1))=h(3)=6 $ \\
									          $\cdot$ \\
									          $\cdot$ \\
									          $\cdot $\\
				$h(g^-1(f^-1(6)))=h(g^-1(2))=h(1)=3 $					
				
				\end{center}
					   

\newpage


 \textbf{2)}

We know that: 
%\begin{align*}


$\epsilon =(\begin{smallmatrix}                   
1 & 2& 3 &4 \\                                                                                                        
1 & 2 & 3& 4
\end{smallmatrix}) $

\medskip

  			         
$f=(\begin{smallmatrix}
1 & 2& 3 &4 \\
2 & 1 & 4 & 3
\end{smallmatrix})$

\medskip



$g=(\begin{smallmatrix}
1 & 2& 3 &4 \\
3 & 4 & 1 & 2
\end{smallmatrix})$


\medskip


$h=(\begin{smallmatrix}
1 & 2& 3 &4 \\
4 & 3 & 2 & 1
\end{smallmatrix})$%\end{align*}

\medskip


then,



\begin{table} [ht]
 \begin{tabular}{ | c | c | c | c | c |c|} 
 	\hline  
 	$\circ$  & $\epsilon$ & f & g & h \\
	\hline 
	$\epsilon$ &  $\epsilon$ & f  &  g &  h\\
	\hline
	f   & f  &  $\epsilon$ & h & g \\
	\hline
	g   &   g &  h & $\epsilon$ & f \\
	\hline
	  h &  h  & g  & f & $\epsilon$ 
	\end{tabular}
\end{table} 

\medskip
Check:


\begin{center}
$\epsilon$  $\circ$ f =$\epsilon$ (f(1))=$\epsilon $(2)=2 \\		
			$\cdot$ \\
			$\cdot$ \\
				\hspace{.3in} =$\epsilon$ (f(4))=$\epsilon$(3)=3 \\
\end{center}

we can apply this kind of idea to f, g and h to show what goes in the table.






\medskip


\bigskip
\bigskip


\textbf{3)}

$f=(\begin{smallmatrix}
1& 2 & 3 & 4 & 5 & 6 \\
2 & 3 & 4 & 1 & 6 & 5
\end{smallmatrix}) $

\medskip

First lets put this into cyclic notation:

\begin{center} 
$(1234)(56)$
\end{center}

\medskip

Now take the powers:

\bigskip

$(1234)(56)$

\bigskip


$((1234)(56))^2=(1234)(56) *(1234)(56)=(13)(24)(5)(6)=(13)(24)$

\bigskip

$((1234)(56))^3=(1432)(56)$

\bigskip

$((1234)(56))^4=(1)$

\bigskip
\bigskip
$\therefore{(1234)(56)}=(13)(24), (1432)(56), (1)$
				%=\{(13)(24), (1432)(56), (1)\} creating sets
				



\bigskip
\bigskip

\bigskip


 \textbf{4} We are given:   $f=(\begin{smallmatrix}
1& 2 & 3 & 4 & 5  \\
2 & 1 & 3 & 4 & 5 
\end{smallmatrix}) $
\hspace {.1in}  $g=(\begin{smallmatrix}
1& 2 & 3 & 4 & 5  \\
1 & 2 & 4 & 5 & 3 
\end{smallmatrix}) $


\newpage

\textbf{5)}
To prove that G is a subgroup of $S_A$:

\bigskip


\hspace{.4in} $\mathbf{1}$ Is it nonempty? Yes!

\medskip

The identity function is the identity if $S_A$:
$\epsilon  \circ f(a)= f\circ \epsilon$.
So the identity permutation fixes a which means e is $\in$ G.


\bigskip

\hspace{.4in} $\mathbf{2}$ Closure? Yes!

\medskip

Let $f, g \in S_A$
then $f \circ g \circ a=f(g(a))=f(a)=a \surd$.

\bigskip


\hspace{.4in} $\mathbf{3}$ Inverse? Yes,

\medskip

Let $f^{-1} \in S_A$
then $f^{-1} \circ f(a)=f^{-1}(f(a))=a=f^{-1}(a) \surd$.



\bigskip




\textbf{6)} We know that it is in $S_9$, so: 

\bigskip


\hspace{.3in}\textbf{a)}
$(145)(37)(682)$             $\mapsto (\begin{smallmatrix}
					1 & 2 & 3 & 4 & 5 & 6 & 7 & 8 \\
					4 & 6 & 7 & 5 & 1 & 8 & 3 & 2
					\end{smallmatrix})$ 
					
					\bigskip
					
\hspace{.3in}\textbf{b)}
$(17)(628)(9354)$             $\mapsto (\begin{smallmatrix}
					1 & 2 & 3 & 4 & 5 & 6 & 7 & 8 \\
					7 & 8  & 5 & 9 & 4  & 2 & 1 & 6
					\end{smallmatrix})$ 
					
					\bigskip

\hspace{.3in}\textbf{c)}
$(71825)(36)(49)$             $\mapsto (\begin{smallmatrix}
					1 & 2 & 3 & 4 & 5 & 6 & 7 & 8 \\
					 8& 5 & 6 & 9 & 7 & 3 & 1 & 2
					\end{smallmatrix})$ 
					
					 \bigskip
\hspace{.3in}\textbf{d)}
$(12)(347)$            		 $\mapsto (\begin{smallmatrix}
					1 & 2 & 3 & 4 & 5 & 6 & 7 & 8 \\
					2 & 1 & 4 & 7 & 5 & 6 & 3 & 8
					\end{smallmatrix})$ 
					
					 \bigskip

\hspace{.3in}\textbf{e)}
$(147)(1678)(74132)$     $\mapsto (\begin{smallmatrix}
					1 & 2 & 3 & 4 & 5 & 6 & 7 & 8 \\
					3 & 8 & 2 & 6 & 5 & 1 & 7 & 4
					\end{smallmatrix})$ 
					
					 \bigskip
					
\hspace{.3in}\textbf{f)}
$(6148)(2345)(12493)$   $\mapsto (\begin{smallmatrix}
					1 & 2 & 3 & 4 & 5 & 6 & 7 & 8 \\
					3 & 5 & 4 & 9 & 2 & 1 & 7 & 6
					\end{smallmatrix})$
					
\bigskip
					
\textbf{7} Permutations in $S_A \mapsto$ disjoint cycles: 

\medskip 


\hspace{.2in}\textbf{a)}$(\begin{smallmatrix}
1 & 2 & 3 & 4 & 5 & 6 & 7 & 8 & 9\\
4 & 9 & 2 & 5 & 1 & 7 & 6 & 8 & 3
\end{smallmatrix})$			
						$\mapsto(145)(293)(67)(8)\mapsto \textit{conventionally written}: (145)(293)(67)$

\bigskip

\hspace{.2in}\textbf{b)}$(\begin{smallmatrix}
1 & 2 & 3 & 4 & 5 & 6 & 7 & 8 & 9\\
7 & 4 & 9 & 2 & 3 & 8 & 1 & 6 & 5
\end{smallmatrix})$			$\mapsto(17)(24)(395)(68)$

\bigskip

\hspace{.2in}\textbf{c)}$(\begin{smallmatrix}
1 & 2 & 3 & 4 & 5 & 6 & 7 & 8 & 9\\
7 & 9 & 5 & 3 & 1 & 2 & 4 & 8 & 6
\end{smallmatrix})$			$\mapsto(17435)(296)$

\bigskip

\hspace{.2in}\textbf{d)}$(\begin{smallmatrix} 
1 & 2 & 3 & 4 & 5 & 6 & 7 & 8 & 9\\
9 & 8 & 7 & 4 & 3 & 6 & 5 & 1 & 2
\end{smallmatrix})$			$\mapsto(1928)(375)$

\bigskip

\bigskip

\textbf{8)}  $\mapsto$ product of Transpositions: 

\bigskip

\hspace{.2in}\textbf{a}  $(137428)$	$\mapsto(18)(12)(14)(17)(13)$

\bigskip

\hspace{.2in}\textbf{b} $(416)(8235)$	$\mapsto(46)(41)(85)(83)(82)$
\bigskip


\hspace{.2in}\textbf{c} $(123)(456)(1574)$	$\mapsto(13)(12)(46)(45)(14)(17)(15)$

\bigskip


\hspace{.2in}\textbf{d)}    \text{We have to transform this into disjoint cycles: }
\begin{center}
$(\begin{smallmatrix}
1 & 2 & 3 & 4 & 5 & 6 & 7 & 8 \\
3 & 1 & 4 & 2 & 8 & 7 & 6 & 5
\end{smallmatrix})$			$\mapsto(1342)(58)(67)$
\end{center}

We we can write it into as a product of transpositions in $S_8$:

\medskip

$(1342)(58)(67) 	\mapsto(12)(14)(13)(58)(67)$


\bigskip

\textbf{9)} Even or Odd?


\medskip

\hspace{.2in}\textbf{a}
$(\begin{smallmatrix}
1 & 2 & 3 & 4 & 5 & 6 & 7 & 8 & \\
7 & 4 & 1 & 5 & 6 & 2 & 3 & 8 &
\end{smallmatrix})$

\medskip

We have to put this into disjoint cycles:

\begin{center}
$(\begin{smallmatrix}
1 & 2 & 3 & 4 & 5 & 6 & 7 & 8 & \\
7 & 4 & 1 & 5 & 6 & 2 & 3 & 8 &
\end{smallmatrix})$		$\mapsto(173)(2456)$.
\end{center}

\medskip

Now we can put the disjoint cycles to product of transpositions which will help us decide if it is even or odd:

\medskip
$(173)(2456)$		$\mapsto(13)(17)(26)(25)(24)$   $\longrightarrow$ this is Odd


\medskip
\bigskip


\hspace{.2in}\textbf{b} \hspace{.1in} $(71864)$		$\mapsto(74)(76)(78)(71)$	$\longrightarrow$  Even

\medskip

\hspace{.2in}\textbf{c} \hspace{.1in} $(123)(456)(1574)$	 $\mapsto(13)(12)(46)(45)(14)(17)(15)$ 		$\longrightarrow$ Odd

\medskip

\hspace{.2in}\textbf{d} \hspace{.1in}  $(1276)(3241)(7812)$ 		$\mapsto (16)(17)(12)(31)(34)(32)(72)(71)(78)$		$\longrightarrow$ Odd

\medskip
 
\textbf{10a)}

Assume that the set T of all transpisitions in $S_n generates S_n$ then,

let $(x_1,x_2 \cdot \cdot \cdot x_n)=(x_1 x_2), (x_3 x_4) \cdot \cdot \cdot (x_n-1 x_n) \in$ T, 
then for example

\begin{center}
the following is in $S_7$:

 (143267)=(17)(16)(12)(13)(14) 
 \end{center}

show that these transpositions generates all of the cycles.

\bigskip

\textbf{10b)}

Assume that the set $T_1=\{(12),(13), . . . (1n)\}  generates  Sn.$

We have $(12)(13)...$, then 

\begin{center}
$(12)=(23) $

\medskip

$(23)=(34)$

and so on
\end{center}

With this $T_1$ will generate to $S_n$ for whatever n is.

\bigskip

For example, let the following generate $S_7$
\bigskip


\begin{center}
\medskip
$(1765432)=(12)(13)(14)(15)(16)(17)$
\end{center}

$\therefore$ As you can see the following transpositions does generate $S_7$.






	

%\[
%h=\big(\begin{smallmatrix} 
%1 & 2 & 3 & 4 & 5 & 6 \\
%3 & 1 & 6 & 4 & 5 & 2
%\end{smallmatrix}\big)
%\]

\end{document}
