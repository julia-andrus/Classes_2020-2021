 \documentclass{article}
\usepackage{ amsmath, amssymb, soul, color, amsthm}
\usepackage{mathtools}
\usepackage{tikz}
\usetikzlibrary{positioning}
\usepackage{soul, color}
\usepackage{ulem}

\title{STAT 461\_Hw 1}
\author{Julia Andrus}
\date{}

\begin{document}

\maketitle 



\textbf{ 1.1-2a)}  (a) all insure at lease one car, (b) at least 85\% insure more than one car, (c) 23\% insure a sports car, and (d) 17\% insure more than one car. P(exactly one car and it is not a sports car)?

\vspace{2mm}
To find out the percentage of people who insure exactly one sports car, we have to subtract (c) and (d):  23\% - 17\% = 6\%

\vspace{2mm}
We can use (a) and (b) to get 15\% that insure exactly 1 car:1-.85=.15
\vspace{2mm}

Using these results, we can subtract them:15\% - 6\% = 9\%
\vspace{2mm}

\hl{Answer:} 9\% insure exactly 1 non-sport car.
%%%%%%%%%%%%%%%%%%%%%%%%%%%%%%%%%
\newpage
\textbf{ 1.1-4a)}  List each of 16 sequences in the sample space:

\vspace{3mm}

S=
$\begin{matrix}

\{HHHH & HHHT & HHTH & HTHH &THHH & HTHT & HHTT & HTTH \\
TTTT & TTTH & TTHT & THTT & HTTT&THTH & TTHH&THHT\}
\end{matrix}$

\vspace{3mm}
\textbf{ 1.1-4b)}  A=3 heads, B=at most 2 heads, C= heads on the third toss, D= 1head and 3 tails

\vspace{3mm}

\textbf{i)} $P(A)=\{HHHT, HHTH, HTHH,THHH, HHHH\}=\frac{5}{16}$
\vspace{2mm}

\hl{Answer:} $\frac{5}{16}$

\vspace{3mm}

\textbf{ii)} P($A\cap B$)=0

\vspace{3mm}

\textbf{iii)} $P(B)=\{TTTH, TTHT, THTT HTTT, THTH, TTHH, TTHH, THHT, HTTH, HHTT, HTHT\}$

\hspace{.58in}$=\frac{11}{16}$
\vspace{2mm}

\hl{Answer:} $\frac{11}{16}$


\vspace{3mm}

\textbf{iv)} $P(A\cap C)=\{HHHH,  HHHT, HTHH, THHH\}=\frac{4}{16}=\frac{1}{4}$
\vspace{2mm}

\hl{Answer:} $\frac{1}{4}$

\vspace{3mm}

\textbf{v)} $P(D)=\{TTTH, THTT,  THTT, TTHT\}=\frac{4}{14}=\frac{1}{4}$
\vspace{2mm}

\hl{Answer:} $\frac{1}{4}$


\vspace{3mm}


\textbf{vi)} $P(A\cup C)=P(A)+P(C)-P(A\cap C)$

\vspace{3mm}

$P(A)=\frac{5}{16}$

\vspace{2mm}

$P(C)=\{HHHH, HHHT,  HTHH, THHH, HTHT, TTHT, TTHH, THHT\}$

\hspace{.36in}$=\frac{8}{16}$

\vspace{2mm}

$P(A\cap C)=\frac{1}{4}=0.25$
\vspace{2mm}


$P(A\cup C)=P(A)+P(C)-P(A\cap C)=\frac{5}{16}+\frac{8}{16}-\frac{1}{4}=.5625$
\vspace{2mm}

\hl{Answer:} $0.5625$


\vspace{2mm}


\textbf{vii)} $P(B\cap D)=P(D)=\{TTTH, THTT,  THTT, TTHT\}=\frac{4}{14}=\frac{1}{4}$
\vspace{2mm}

\hl{Answer:} $\frac{1}{4}$



%%%%%%%%%%%%%%%%%%%%%%%%%%%%%%%%%
\newpage

\textbf{1.1-14)} Picture below shows the solution to problem:


\includegraphics[width=0.9\columnwidth]{../IMG_0532.HEIC.pdf}



%%%%%%%%%%%%%%%%%%%%%%%%%%%%%%%%% 1.1-16, 1.2-2, 1.2-4, 1.2-6, 1.2-10, 1.2-12, 1.2-16, 1.3-2, 1.3-4, 1.3-8, 1.3-10, 1.3-12, 1.4-2, 1.4-4, 1.4-10, 1.4-18, 1.5-6, 1.5-10
\newpage

\textbf{1.1-16)} $P_{n}, n=0,1,2... $ be a the probability that automobile policyholder will file for n claims in 5 years period. Actuary making assumption: $P_{n+1}=
 (\frac{1}{4})_{p}.$ What is the probability that the holder will file two or more claims during the period.
 
  \vspace{2mm}
 
 We have to bring in the idea of disjoint events:
 \vspace{2mm}
 
$P(n\ge2)$=(\text{for disjoint events (using the addition rule}))

\hspace{.55in}$=p_{2}+p_{3}+......+p_{n}$

\hspace{.55in}$=\frac{1}{4}p_{1}+\frac{1}{4}p_{2}+......+\frac{1}{4}p_{n}$

\hspace{.55in}$=\frac{1}{16}p_{0}+\frac{1}{16}p_{1}+......+\frac{1}{16}p_{n}$

\hspace{.55in}$=\frac{1}{16}(p_{0}+p_{1}+......+p_{n})$

\vspace{4mm}
$\frac{1}{16}(p_{0}+p_{1}+......+p_{n})=1$ because it's 0 or more claims file, so:
\vspace{2mm}


$\frac{1}{16}(p_{0}+p_{1}+......+p_{n})=\frac{1}{16}(1)$
 \vspace{2mm}
$=\frac{1}{16}(1)$

\hspace{1.43in}$=\frac{1}{16}$

\hl{Answer:} $\frac{1}{16}$



%%%%%%%%%%%%%%%%%%%%%%%%%%%%%%%%%  1.2-2, 1.2-4, 1.2-6, 1.2-10, 1.2-12, 1.2-16, 1.3-2, 1.3-4, 1.3-8, 1.3-10, 1.3-12, 1.4-2, 1.4-4, 1.4-10, 1.4-18, 1.5-6, 1.5-10
\newpage

\textbf{1.2-2a)} 4 levels of temp, 5 different pressures, 2 different catalysts. Order does matter, but repetition is okay. Consider all possible combinations, how many experiments would need to be conducted?

\vspace{4mm}

Since there are 4 levels of temp, 5 different pressures, 2 different catalysts, we can just multiply all these together:
\vspace{4mm}

We know we are dealing with permutation:
\vspace{4mm}


$${4}\times {5}\times {2}=40$$

\hl{Answer:} So 40 combinations

\vspace{5mm}

\textbf{1.2-2b)} 3 factors, 2 levels.

temp: 2 ways

pressure: 2 ways 


catalysts: 2 ways

So,

 $$2\times2\times2=8$$. 
 
 \hl{Answer}: There are 8 combinations.




%%%%%%%%%%%%%%%%%%%%%%%%%%%%%%%%%  1.2-4, 1.2-6, 1.2-10, 1.2-12, 1.2-16, 1.3-2, 1.3-4, 1.3-8, 1.3-10, 1.3-12, 1.4-2, 1.4-4, 1.4-10, 1.4-18, 1.5-6, 1.5-10
\newpage

\textbf{1.2-4a)} We have 4 ice cream flavors, 6 toppings. How many sundaes are possible using 1 flavor of ice cream and 3 toppings?
\vspace{2mm}

We are going to use combination formula: 

$$4\times_{n}C_{r}=4\times \frac{n!}{r!(n-r)!}$$

\vspace{2mm}
We are multiplying by 4 because of the 4 different flavors:

$$\hspace{.2in}4\times_{6}C_{3}=4\times \frac{6!}{3!(6-3)!}$$
$$=80$$

\hl{Answer:} 80 combinations are possible with 1 flavor and 3 toppings.

%%%%%%%%%%
\vspace{2mm}
\textbf{1.2-4b )} How many sundaes are possible using 1 flavor of ice cream and from 0-6 toppings?

$$4\times 2^{6}=256$$

\hl{Answer:} 256

\vspace{3mm}

%%%%%%%%%%
\textbf{1.2-4c )} Flavors of 3 scoops of ice cream are possible if it is permissible to make all three scoops the same flavor.

\vspace{2mm}

Doing diagrams help me figure out what is going on here:

\vspace{2mm}

$\begin{matrix}
111 & 112 & 113 & 114 & 122 & 123 & 124 & 133 & 134 & 144 \\
222 & 223 & 224 & 233 & 234 & 244\\
333  & 334 & 344\\
444
\end{matrix}$
\vspace{2mm}

With this, we need to use the combination formula: n=4, r=3

$$(n-1+r)C_{r}=\frac{(n-1+r)!}{r!(n-1)!}$$
$$(4-1+3)C_{3}=\frac{(4-1+3)!}{3!(4-1)!}$$
$$\hspace{.2in}_6C_{3}=\frac{(6)!}{3!(3)!}$$
$$\hspace{.45in}=\frac{(6)!}{3!(3)!}$$
$$\hspace{.25in}=20$$

\hl{Answer:} There are 20 different possible combinations.




%%%%%%%%%%%%%%%%%%%%%%%%%%%%%%%%%  1.2-6, 1.2-10, 1.2-12, 1.2-16, 1.3-2, 1.3-4, 1.3-8, 1.3-10, 1.3-12, 1.4-2, 1.4-4, 1.4-10, 1.4-18, 1.5-6, 1.5-10
\newpage

\textbf{1.2-6)}  First player win 3 sets wins the match. 
\vspace{2mm}

One way could be 3x0 D or F: \{DDD, FFF\}. This gives us 2 ways.
\vspace{2mm}

Another way could be 3x1 D or F: \{DFDD , FDFF, DDFD , FFDF, FDDD , DFFF\}. This give us 6 ways.
\vspace{2mm}

Another way could be 3x2 D or F: \{DFDFD , FDFDF ,DDFFD , FFDDF, DFFDD , FDDFF, FFDDD , DDFFF, FDFDD,   DFDFF, FDDFD,  DFFDF\}. This gives us 12 ways. 
\vspace{2mm}

Adding all the ways together: 2 + 6 + 12=20. 

\vspace{2mm}
\hl{Answer:} So there are 20 ways the tennis match could end.

%%%%%%%%%%%%%%%%%%%%%%%%%%%%%%%%%   1.2-10, 1.2-12, 1.2-16, 1.3-2, 1.3-4, 1.3-8, 1.3-10, 1.3-12, 1.4-2, 1.4-4, 1.4-10, 1.4-18, 1.5-6, 1.5-10
\newpage

\textbf{1.2-10)} Prove Pascal's equation work:

\vspace{2mm}
Prove  $(_{r}^{n})=(_{r}^{n-1}) + (_{r-1}^{n-1})$


\begin{proof}
$$(_{r}^{n-1}) + (_{r-1}^{n-1})=\frac{(n-1)!}{r!(n-r-1)!} + \frac{(n-1)!}{(r-1)!(n-r)!}$$
$$\hspace{.5in}=\frac{(n-1)! (n-r) + (n-1)! r}{r!(n-r)!}$$
$$\hspace{.15in}=\frac{((n-r)+r)(n-1)!}{r!(n-r)!}$$
$$\hspace{-.5in}=\frac{n(n-1)!}{r!(n-r)!}$$
$$\hspace{-.5in}=\frac{(n)!}{r!(n-r)!}$$
$$\hspace{-.99in}\text{=}(^{n}_{r})$$
\end{proof}




%%%%%%%%%%%%%%%%%%%%%%%%%%%%%%%%%    1.2-12, 1.2-16, 1.3-2, 1.3-4, 1.3-8, 1.3-10, 1.3-12, 1.4-2, 1.4-4, 1.4-10, 1.4-18, 1.5-6, 1.5-10
\newpage

\textbf{1.2-12)}  To be able to proof both of these we need to know the Binomial Theorem: 


$$(x+y)^{n}=\sum_{j=0}^{n}(_{j}^{n}) x^{n-j} y^{j}$$

Now we can start to proof:
\vspace{2mm}

Prove: $\sum_{r=0}^{n}(1-r)^{n}(^{n}_{r})=0$

\begin{proof}
$$0=0^{n}$$
$$\hspace{.45in}=(1-1)^{n}$$
$$\hspace{.65in}=(1+(-1))^{n}$$

\text{Starting here is when we can use the Binomial Theorem: x=1, y=-1}

$$\hspace{.3in}=\sum_{r=0}^{n}(_{r}^{n})1^{n-r} (-1)^{r}$$
$$=\sum_{r=0}^{n} (-1)^{r} (_{r}^{n})$$
\end{proof}

\vspace{3mm}


Prove: $\sum_{r=0}^{n}(^{n}_{r})=2^{n}$

\begin{proof}
$$2^{n} = (1+1)^{n}$$

\text{Again, here is when we can use the Binomial Theorem: x=1, y=1}
\vspace{2mm}
$$=\sum_{r=0}^{n} (_{r}^{n}) 1^{n-r} (1)^{r}$$
$$\hspace{-.5in}=\sum_{r=0}^{n} (_{r}^{n})$$
\end{proof}



%%%%%%%%%%%%%%%%%%%%%%%%%%%%%%%%%   1.2-16, 1.3-2, 1.3-4, 1.3-8, 1.3-10, 1.3-12, 1.4-2, 1.4-4, 1.4-10, 1.4-18, 1.5-6, 1.5-10
\newpage

\textbf{1.2-16a)} 52 hearts: 19 white, 10 tan, 7 pink, 3 purple, 5 yellow. 9 pieces randomly collected. What is the probability that 3 hearts are whites?

Using this formula: $(^{n}_{r})=\frac{n!}{r!(n-r)!}$

$$\hspace{-2.1in}P( \text{3 hearts are whites})=\frac{(^{19}_{3})(^{33}_{6})}{^{52}_{9}}$$

$$=\frac{\frac{19!}{3!(19-3)!} \times \frac{33!}{6!(19-6)!}}{(^{52}_{9})}$$
$$\hspace{-.9in}=0.29$$

\hl{Answer:} $0.29$

\vspace{2mm}
%%%%%%%%%%%%

\textbf{1.2-16b)} Prob. of 3 white, 2 tan, 1 pink, 1 yellow, 2 green

P(3 white, 2 tan, 1 pink, 1 yellow, 2 green)=

$$\frac{(^{19}_{3})(^{10}_{2})(^{7}_{1})(^{5}_{1}(^{6}_{2})}{(^{52}_{9})}=0.0062$$


\hl{Answer:} $0.0062$

%%%%%%%%%%%%%%%%%%%%%%%%%%%%%%%%%    1.3-2, 1.3-4, 1.3-8, 1.3-10, 1.3-12, 1.4-2, 1.4-4, 1.4-10, 1.4-18, 1.5-6, 1.5-10
\newpage

\textbf{1.3-2a)} 1456 people by gender and by whether or not they favor a gun law. What is $P(A_{1})?$

$$P(A_{1})=\frac{1041}{1456}$$

$$\hspace{.4in}= 0.715$$

\hl{Answer:} $00.715$

\vspace{2mm}


%%%%%%%%%%%%%%%%%%%
\textbf{1.3-2b)} $P(A_{1}| S_{1})$?

$$P(A_{1}| S_{1})=\frac{P(A_{1}\cap S_{1}}{P(S_{1})}$$
$$\hspace{.25in}=\frac{392}{633}$$
$$\hspace{.4in}= 0.6193$$

\vspace{2mm}
\hl{Answer:} 0.6193

\vspace{2mm}
%%%%%%%%%%%%%%%%%%%
\textbf{1.3-2c)} $P(A_{1}| S_{2})$? 

 $$P(A_{1}| S_{2})=\frac{P(A_{1}\cap S_{2}}{P(S_{2})}$$
$$\hspace{.2in}=\frac{649}{823}$$
$$\hspace{.35in}= 0.7886$$

\hl{Answer:} 0.7886

\vspace{5mm}

\textbf{1.3-2d)} interpret (a) and (b). 

\vspace{2mm}

Part (a) is asking for the probability of $A_{1}$ given $S_{1}$ and part (b) is asking for the probability of $A_{1}$ given $S_{2}$. This also shows that 78.86\% of females are in favor of gun law and 61.93\% of males are in favor of gun law.


%%%%%%%%%%%%%%%%%%%%%%%%%%%%%%%%%  1.3-4, 1.3-8, 1.3-10, 1.3-12, 1.4-2, 1.4-4, 1.4-10, 1.4-18, 1.5-6, 1.5-10
\newpage

\textbf{1.3-4a)} Two hearts. Without replacement from an ordinary deck of playing cards. 
\vspace{2mm}

We know that there are 52 cards in a deck. We also know that there are 13 aces, 13 heart, 13 spades, and 13 clubs. So P(1st heart) $\frac{13}{52}$ cards are hearts. When 1 heart is taken, now  the  P(2nd heart)=$\frac{12}{51}$ hearts left. 

To get P(two hearts), we have to multiply our P(1st heart) and P(2nd heart):

$$\hspace{-.2in}P(\text{two hearts})=P(\text{1st, heart} \cap \text{2nd, heart})$$
$$\hspace{.9in}=P(\text{1st, heart}) \times P(\text{2nd,  heart})$$
$$\hspace{-.35in}=\frac{13}{52} \times \frac{12}{51}$$
$$\hspace{-.7in}=\frac{1}{17}$$

\hl{Answer:} $\frac{1}{17}$

\vspace{5mm}

%%%%%%%%%%%%%%%%%%%

\textbf{1.3-4b)} A heart on the first draw and ace on the second draw.

Drawing heart first: $\frac{13}{52} $hearts. For drawing ace second we have 13 clubs, but now we have 51 cards remaining because we drew a card (heart): $\frac{13}{51}$
\vspace{2mm}
Same idea as part (a):

$$\hspace{-1.1in}P(\text{ 1st, heart and 2nd, club})=P(\text{1st, heart} \cap \text{2nd, club})$$
$$\hspace{.9in}= P(\text{1st, heart}) \times P(\text{2nd, club})$$
$$\hspace{-.35in}= \frac{13}{52} \times \frac{13}{51}$$
$$\hspace{-.6in}=\frac{13}{204}$$

\hl{Answer:} $ \frac{13}{204}$


%%%%%%%%%%%%%%%%%%%
\vspace{5mm}

\textbf{1.3-4c)} Heart on first draw can be drawn by getting the ace of hearts or one of the other 12 hearts. 
\vspace{2mm}

The probability of me drawing an ace of hearts is $\frac{1}{52}$ and probability of hearts without ace of hearts is $\frac{12}{52}$ . Now there are $\frac{3}{51}$ aces left.  Now there are 4 aces left after we have selected hearts that are not ace of hearts: $\frac{4}{51}$. 

\vspace{2mm}

With these, we can multiply all these combinations together:

$$\hspace{.45in}= \frac{1}{52} \times \frac{12}{52} \times \frac{3}{51} \times \frac{4}{51} \times \frac{13}{51}$$
$$\hspace{-.9in}= \frac{1}{52}$$

\hl{Answer:} $\frac{1}{52}$

%%%%%%%%%%%%%%%%%%%%%%%%%%%%%%%%% 1.3-8, 1.3-10, 1.3-12, 1.4-2, 1.4-4, 1.4-10, 1.4-18, 1.5-6, 1.5-10
\newpage

\textbf{1.3-8a)} If you draw first, find the prob. that you win in the game on your second draw. 


17 balls marked L, 3 balls marked W. We have to add these together gives us our n value: 17+3=20

$$P(W_{1}=\frac{\text{\# of favorable outcomes }}{\text{\# of possible outcomes}}=\frac{3}{20}$$

There is still 2 more W balls:

$$P(W_{2}|P(W_{1}))=\frac{\text{\# of favorable outcomes }}{\text{\# of possible outcomes}}=\frac{2}{19}$$

Now there is 1 more W ball:

$$P(W_{3}| W_{1}\cap W_{2})=\frac{\text{\# of favorable outcomes }}{\text{\# of possible outcomes}}=\frac{1}{18}$$

Now multiply all this together to get our answer:

$$P(W_{1}\cap W_{2} \cap W_{3})=P(W_{1}) \times P(W_{2}) \times P(W_{3})$$
$$\hspace{.4in}=\frac{3}{20} \times \frac{2}{19} \times \frac{1}{18}$$
$$\hspace{-.15in}=\frac{1}{1140}$$


\hl{Answer:} $\frac{1}{1140}$


\vspace{2mm}
%%%%%%%%%%%%%%%%%%%

\textbf{1.3-8b)} If you draw first, what is the prob. that your opponent wins the game one his second draw.

First we need to find what the prob. of 2 W and 1 L and use the combination formula, 
$$P(\text{2 W and 1 L})=\frac{_{3}C_{2}\times_{17}C_{1}}{_{20}C_{3}}=\frac{3\times17}{1140}=\frac{17}{380}$$. 

There is still one more W ball we have to account for: $\frac{1}{17}$

To get our final answer we have to multiply multiply these all together:

$$P(\text{2 W and 1 L} \cap \text{1 W ball remaining})=\frac{17}{380} \times \frac{1}{17}$$


\hl{Answer:} $= 0.002632$

\vspace{5mm}
\newpage
%%%%%%%%%%%%%%%%%%%
\textbf{1.3-8c)} If you draw first, what is the prob. that you will win?
\vspace{2mm}

Since I have ten draws, I would have to find the prob of me wining on second draw, on third draw....on tenth draw:


$$P(\text{wining on 2nd draw})=\frac{1}{1140}\text{(answer from part a)}$$
$$\hspace{-.5in}P(\text{wining on 3rd draw})=\frac{_{3}C_{2}\times_{17}C_{4}}{_{20}C_{6}}\times \frac{1}{16}$$
$$\hspace{-.5in}P(\text{wining on 4th draw})=\frac{_{3}C_{2}\times_{17}C_{6}}{_{20}C_{8}}\times \frac{1}{14}$$
$$\hspace{-.5in}P(\text{wining on 5th draw})=\frac{_{3}C_{2}\times_{17}C_{8}}{_{20}C_{10}}\times \frac{1}{12}$$
$$\hspace{-.1in}.$$
$$\hspace{-.1in}.$$
$$\hspace{-.1in}.$$
$$\hspace{-.45in}P(\text{wining on 10th draw})=\frac{_{3}C_{2}\times_{17}C_{16}}{_{20}C_{18}}\times \frac{1}{2}$$

After a long process of getting these results, we have to add all these together to get our P(winning):

$$P(\text{winning})=\frac{1}{1140} + \frac{1}{190} + \frac{1}{76} + \frac{7}{285} + \frac{3}{76} + \frac{11}{190} + \frac{91}{1140} + \frac{2}{19} + \frac{51}{380} $$


\hl{Answer:} $=0.4605$

%%%%%%%%%%%%%%%%%%%
\vspace{5mm}

\textbf{1.3-8d)} Would you prefer to draw first or second? Why?
\vspace{2mm}

Second, because the prob. of the first person who draw wins is $0.4605$, which is less than half. If you draw second, than you would win.  




%%%%%%%%%%%%%%%%%%%%%%%%%%%%%%%%% 1.3-12, 1.4-2, 1.4-4, 1.4-10, 1.4-18, 1.5-6, 1.5-10
\newpage

\textbf{1.3-10a)} Single card is drawn at random from each of well shuffled decks of playing cards. Let A be the event that all six cards drawn are different. P(A)?
\vspace{2mm}

There are 52 cards in a deck and there are 6 decks of cards, so n=52. With the 6 decks of cards we  can draw 52 cards from the first 1st deck: $P(D_{1})=\frac{52}{52}$. Since we already selected one card, now there is 51 cars left that we can draw from second set: $P(D_{2}|D_{1})=\frac{51}{52}$. We can keep doing this with 3, 4, 5, 6. At the end we would get 

$$P(D_{1})=\frac{52}{52}$$
$$P(D_{2})=\frac{51}{52}$$
$$P(D_{3})=\frac{50}{52}$$
$$P(D_{4})=\frac{49}{52}$$
$$P(D_{5})=\frac{48}{52}$$
$$P(D_{6})=\frac{47}{52}$$

Now, to get P(A) we have to multiply all these together:
$$\hspace{.6in}P(A)=P(D_{1})\cap(D_{2})\cap(D_{3})\cap(D_{4})\cap(D_{5})\cap(D_{6})$$
$$\hspace{.3in}= \frac{52}{52}\times \frac{51}{52}\times \frac{50}{52}\times \frac{49}{52}\times \frac{48}{52}\times \frac{47}{52}$$
$$\hspace{-.9in}=\frac{8,808,975}{11,881,376}$$

\hl{Answer:} 0.741

%%%%%%%%%%%%%
\vspace{5mm}


\textbf{1.3-10b)} Find the prob. that at least  2 of the drawn cards match.

We have to use the complementary prob. formula: $P(A^{c})=1-P(A)$

$$1-P(A)=1-\frac{8,808,975}{11,881,376} = 0.2586$$

 \hl{Answer:} 0.2586
 
 
 %%%%%%%%%%%%%%%%%%%%%%%%%%%%%%%%% 1.3-12, 1.4-2, 1.4-4, 1.4-10, 1.4-18, 1.5-6, 1.5-10
\newpage

\textbf{1.3-12a)} 18 students. 18 chips: 1 blue and 17 red. Without replacement. If you have a choice of drawing first, fifth, or last, which position would you choose?

\vspace{2mm}
Position wouldn't matter, because of the definition of combinations (order does not matter). Different orders of choosing would lead to the same chips getting selected. You would have the same result if you chose to go first, fifth, or last.
\vspace{2mm}


If you went first, then your probability of your choice is $\frac{1}{18}$. 
\vspace{2mm}

If you chose to go fifth then 4 people have already chose their options, which means you have 14 chips left to choose from: $\frac{_{17}C_{4}}{_{18}C_{4}}\times \frac{1}{14}=\frac{7}{9}\times\frac{1}{14}=\frac{1}{18}$

\vspace{2mm}
If you chose to go last then 17 people chose before you, which leaves you 1 chip: $\frac{_{17}C_{17}}{_{18}C_{17}}\times 1 =\frac{1}{18}\times1=\frac{1}{18}$

\vspace{2mm}


 %%%%%%%%%%%%%%%%%%%%
 \vspace{5mm}
\textbf{1.3-12ab} Suppose the bowl contains 2 blue chips and 16 red chips, which position would you choose? 

 \vspace{2mm}
For this problem, position also does not matter. Going through the same process as part (a), they would all of a $\frac{1}{9}$ chance of getting the A. 



 %%%%%%%%%%%%%%%%%%%%%%%%%%%%%%%%% 1.4-2, 1.4-4, 1.4-10, 1.4-18, 1.5-6, 1.5-10
\newpage

\textbf{1.4-2a)} Let $ P(A) =0.3$ and $P(b)=0.6.$ Find $P(A\cup B)$ when A and B are independent events. 

$$P(A\cup B)=P(A)+P(B)-P(A\cap B)$$
$$\hspace{.4in}=0.3+0.6-(0.3\times 0.6)$$


\hl{Answer:} 0.72

\vspace{3mm}

\textbf{1.4-2b)} $P(A | B)$ when A and B are mutually exclusive.
\vspace{3mm}

Using this formula: $P(A|B)=\frac{P(A \cap B)}{P(B)}$ 
\vspace{3mm}

$P(A \cap B)=0$ because A and B does not exist. So,

$$P(A|B)=\frac{P(A \cap B)}{P(B)}$$
$$\hspace{.25in}=\frac{0}{P(B)}$$

\hl{Answer:} 0


 %%%%%%%%%%%%%%%%%%%%%%%%%%%%%%%%%  1.4-4, 1.4-10, 1.4-18, 1.5-6, 1.5-10
\newpage

\textbf{1.4-4)} Prove part (a) and (c) from theorem 1.4-1

Theorem 1.4-1 (b) says: A and B are independent events, proof $A^{'}$ and B are independent events.

\begin{proof}


\text{Multiplication rule}

 $$\hspace{-.8in}P(A^{'}\cap B)=P(B)P(A^{'} | B)$$

 \text{Complement rule}
 
$$\hspace{.1in}=P(B)(1-P(A | B)) $$

\text{Conditional Prob. rule}

$$\hspace{.2in}=P(B)(1-\frac{P(A\cap B)}{P(B)})$$

\text{Multiplying independent events}

$$\hspace{.2in}=P(B)(1-\frac{P(A)P( B)}{P(B)}) $$
$$\hspace{-.2in}=P(B)(1-P(A))$$
$$\hspace{-.5in}=P(B)P(A^{'})$$
\text{swap}
$$\hspace{-.5in}=P(A^{'})P(B)$$
\end{proof}

\vspace{2mm}

Theorem 1.4-1 (c) says: A and B are independent events, proof $A^{'} \text{and} B^{'}$ are independent events.

\begin{proof}
\text{Multiplication rule}

$$\hspace{-1in}P(A^{'}\cap B^{'})=P(B^{'})P(A^{'} | B^{'})$$

\text{Complement rule}
$$\hspace{.3in}=(1-P(B))(1-P(A | B^{'}))$$

\text{Conditional Prob. rule}

$$\hspace{.4in}=(1-P(B))(1-\frac{P(A\cap B^{'})}{P(B^{'})})$$

$$\hspace{.5in}=(1-P(B))(1-\frac{P(A) P(B^{'})}{P(B^{'})})$$

$$\hspace{.1in}=(1-P(B))(1-P(A))$$
\text{swap}
$$\hspace{-.3in}=P(B^{'})P(A^{'})$$

$$\hspace{-.3in}=P(A^{'})P(B^{'})$$
\end{proof}

 %%%%%%%%%%%%%%%%%%%%%%%%%%%%%%%%%  1.4-10, 1.4-18, 1.5-6, 1.5-10
\newpage

\textbf{1.4-10a)} Let $D_{1}, D_{2}, D_{3}$ be 3 four sided dice whose sides have been labeled as follows: 
\vspace{2mm}

$D_{1}$: 0333
\vspace{2mm}

$D_{2}$: 2225

\vspace{2mm}
$D_{3}$:1146

\vspace{2mm}
Show $P(A)=\frac{9}{16}$

\vspace{2mm}



Since $D_{1} \ge D_{2}$,  I role a 3 on $D_{1}$ and 2 on the $D_{2}$: $P(D_{1}=3)=\frac{3}{4}\text{ and} P(D_{2}=2)={3}{4} $
\vspace{2mm}

Because the dice are independent, we can multiply these two together:
$$P(A)=\frac{3}{4} \times \frac{3}{4}=\frac{9}{16} $$
\vspace{2mm}


%%%%%%%%%%%%%%%%%%%%
\textbf{1.4-10b)}  Show $P(B)=\frac{9}{16}$.
\vspace{2mm}

Since $D_{2} \ge D_{3}$, then:
\vspace{2mm}
 
$P(D_{2}=5)=\frac{1}{4} \text{and} P(D_{3}=1)={1}{2} \text{and} P(D_{3}=4)={1}{4} $
\vspace{2mm}

Multiplying the independent events:$P(D_{2}=5) \times P(D_{3}=4)$, I get $\frac{1}{16}$ 
\vspace{2mm}

Now, 
\vspace{2mm}

$$P(D_{3}=1)+P(D_{2}=5 \cap D_{3}=4)$$

$$\frac{1}{2}+\frac{1}{16}=\frac{6}{16}$$

%%%%%%%%%%%%%%%%%%%%
\textbf{1.4-10c)}  Show $P(C)=\frac{10}{16}$.
\vspace{2mm}

Since $D_{2} \ge D_{3}$, then:
\vspace{2mm}

$P(D_{1}=0)=\frac{1}{4} \text{and} P(D_{1}=3)={3}{4} \text{and} P(D_{3}=4  \text{or}  6)={1}{2} $

\vspace{2mm}
Multiplying the independent events:$P(D_{1}=3) \times  P(D_{3}=4  \text{or}  6)$, I get $\frac{3}{8}$ 
\vspace{2mm}

Now, 

$$P(D_{1}=0)+P(D_{1}=3) \cap D_{3}=4  \text{or} 6) $$

$$\frac{1}{4}+\frac{3}{8}=\frac{10}{16}$$




 %%%%%%%%%%%%%%%%%%%%%%%%%%%%%%%%% 1.4-18, 1.5-6, 1.5-10
\newpage


\textbf{1.4-18a)} 8 team single-elimination tournament is set. How many coin flips are required to determine the tournament winner?

\vspace{2mm}

We see that there are 4 games going on in the first set: A match B, A match B, C match D, E match F, and G match H. 
\vspace{2mm}

4 games in first set + 2 games in the second set ( 2 winners play each other) + 1 game for the final match.

\vspace{2mm}

\hl{Answer:} Adding all these together, we get 7. 

\vspace{5mm}


 %%%%%%%%%%%%%%
 
\textbf{1.4-18b)} Prob. that you can predict all the winners?
\vspace{2mm}

There are 2 teams in each game, and I have $\frac{1}{2}$\% chance to predict winner of the game. We have to let these games be independent and use multiply these events:

\vspace{2mm}
$$P(B_{1}\cap B_{2}\cap B_{3}\cap B_{4}\cap B_{5}\cap B_{6}\cap B_{7})=P(B_{1}) P(B_{2}) P(B_{3}) P(B_{4}) P(B_{5}) P(B_{6}) P(B_{7})$$
$$\hspace{.3in}=(\frac{1}{2}) (\frac{1}{2}).........(\frac{1}{2})$$
\vspace{2mm}

\hl{Answer} $\frac{1}{28}$

\vspace{5mm}

%%%%%%%%%%%%%%
 
\textbf{1.4-18c)} How many games are required to determine the national champion? 
\vspace{2mm}

First set we have to cut 64 teams into 32 pairs with each pair playing a game. Second set, the 32 that wins from the first set will be cut into 16 pairs and will play a game. This goes on until the 2 winners play the final game. So, 32 +16 + 8+4+2+1=63. 

\vspace{3mm}
There is 63 games played total to determine who the champion is. 


%%%%%%%%%%%%%%
\vspace{5mm}

 
\textbf{1.4-18d)} Same set up as (b), but predicting all 63:


$$P(B_{1}\cap B_{2}\cap B_{3}......\cap B_{63})=P(B_{1}) P(B_{2}) P(B_{3})..... P(B_{63})$$
$$\hspace{.85in}=(\frac{1}{2}) (\frac{1}{2}).........(\frac{1}{2})$$
$$\hspace{.2in}=(\frac{1}{2})^{63}$$
\vspace{2mm}

\hl{Answer:} The answer is no because the probability is way smaller: $\approx 1.0842\times 10^{-19}$

\vspace{2mm}


 
  %%%%%%%%%%%%%%%%%%%%%%%%%%%%%%%%% 1.5-6, 1.5-10
\newpage

\textbf{1.5-6)} What are the conditional probabilities of the deceased having had a standard, a preferred, and an ultra-preferred policy?
\vspace{3mm}

For this problem, let S=standard, PF=preferred, UPF=ultra-preferred, and D=deceased. Have to use Bayes' Theorem to solve this:
\vspace{3mm}

$$P(S | D) = \frac{P(S)P(D | S)}{P(S)P(D | S) + P(PF)P(D | PF) + P(UPF)P(D | UPF)}$$
$$=\frac{(0.60)(0.01)}{(0.60)(0.01) + (0.30)(0.008) + (0.10)(0.007)}$$
$$=\frac{60}{60 + 24 + 7}$$
\vspace{3mm}

\hl{Answer:} 0.659
\vspace{3mm}


$$P(PF | D)= \frac{P(PF)P(D | PF)}{P(S)P(D | S) + P(PF)P(D | PF) + P(UPF)P(D | UPF)}$$
$$P(PF| D) = \frac{(0.30)(0.008)}{(0.60)(0.01) + (0.30)(0.008) + (0.10)(0.007)}$$
$$= \frac{24}{91}$$
\vspace{3mm}

\hl{Answer:} 0.264
\vspace{3mm}

$$P(UPF | D) =\frac{P(UPF)P(D | UPF)}{P(S)P(D | S) + P(PF)P(D | PF) + P(UPF)P(D | UPF)}$$
$$P(UPF | D) = \frac{(0.10)(0.007)}{(0.60)(0.01) + (0.30)(0.008) + (0.10)(0.007) }$$
$$= \frac{7}{91}$$


\hl{Answer:}  0.077




 %%%%%%%%%%%%%%%%%%%%%%%%%%%%%%%%% 1.5-10
\newpage

\textbf{1.5-10a)}  Probability that the doctor classifies this child as abused? Compute: 

\vspace{2mm}

$P(D^{+})=P(A^{+}) P(D^{+} | A^{+}) +P(A^{-}) P(D^{+} | A^{-})$ 

\vspace{2mm}

We are given that:

\vspace{2mm}

$P(A^{+}=0.02$

$P(A^{-}=0.98$

$P(D^{-} | A^{+}=0.08$

$P(D^{+} | A^{-}=0.05$


$P(D^{+} | A^{+}=0.92$


$P(D^{-} | A^{-}=0.95$


\vspace{2mm}
Now we can start solving this: 

\vspace{2mm}

$$P(D^{+})=P(A^{+}) P(D^{+} | A^{+}) +P(A^{-}) P(D^{+} | A^{-})$$

$$\hspace{-.7in} P(D^{+})=(0.02) (0.92) +(0.98) (0.05)$$


\hl{Answer:} 0.0674


\vspace{2mm}

 %%%%%%%%%%%%%%%%

\textbf{1.5-10b)} Using Bayes' Theorem to Compute: $P(A^{-} | D^{+})  \text{and} P(A^{+} | D^{+})$:

For $P(A^{-} | D^{+})$:

$$P(A^{-} | D^{+})=\frac{P(A^{-}) P(D^{+} | A^{-})}{ P(A^{+}) P(D^{+} | A^{+}) + P(A^{-}) P(D^{+} | A^{-})}$$
$$=\frac{0.98(0.05)}{(0.02)(0.92)+(0.98)(0.05)}$$


\hl{Answer:}0.727

\vspace{2mm}

For $P(A^{+} | D^{+})$:

$$P(A^{+} | D^{+})=\frac{P(A^{+}) P(D^{+} | A^{+})}{ P(A^{+}) P(D^{+} | A^{+}) + P(A^{-}) P(D^{+} | A^{-})}$$
$$=\frac{0.02(0.92)}{(0.02)(0.92)+(0.98)(0.05)}$$


\hl{Answer} 0.273

\vspace{2mm}

%%%%%%%%%%%%%%%%

\textbf{1.5-10c)} Using Bayes' theorem again to compute:  $P(A^{-} | D^{-})$ and $P(A^{+} | D^{-})$


 For $P(A^{-} | D^{-})$:
 
$$P(A^{-} | D^{-})=\frac{P(A^{-}) P(D^{-} | A^{-})}{ P(A^{+}) P(D^{-} | A^{+}) + P(A^{-}) P(D^{-} | A^{-})}$$
$$=\frac{0.98(0.95)}{(0.02)(0.08) +(0.98)(0.95)}$$

\hl{Answer:} 0.9983

\newpage

 For $P(A^{+} | D^{-})$:
 
 $$P(A^{+} | D^{-})=\frac{P(A^{+}) P(D^{-} | A^{+})}{ P(A^{+}) P(D^{-} | A^{+}) + P(A^{-}) P(D^{-} | A^{-})}$$
 $$\hspace{-.3in}=\frac{0.02(0.08)}{0.02(0.08) +0.98(0.95)}$$
 $$\hspace{-1.1in}=0.0001716$$
 
 \hl{Answer:} 0.0001716
 
 \vspace{3mm}
 \textbf{1.5-10d)} Yes is it. 






\end{document}
