\documentclass{article}
\usepackage{amssymb, amsmath, graphicx}
\usepackage{pgfplots}
 \usepackage{tikz, calc}
 \usepackage{pst-plot}
 \usetikzlibrary{calc,patterns,angles,quotes}
 \usepackage{kpfonts}
\usepackage{stackengine}
\usepackage{calc}
\newlength\shlength
 \usepackage{kpfonts}
 \usepackage[utf8]{inputenc}
 \usepackage{graphicx}
 
 \usepackage{stackengine}
\title{Math 338\:Write up\_4}
\author{Julia Andrus}
\date{}

\begin{document}

\maketitle

%\includegraphics[width=0.7\columnwidth]{../Screen Shot 2021-01-26 at 12.29.33 PM.png} 

\textbf{Note:} Hong Le and I worked together on some of these problems.

\vspace{2mm}

\textbf{1)} Explore Euclid’s proof of the Pythagorean Theorem using this geogebra sketch: https://ggbm.at/cabbzzja.  Justify each of the steps in the proof.

\vspace{2mm}

\textbf{1) Step: B,A,H are collinear and so are C, A, G}

\vspace{2mm}

When I had no power I wrote out my thoughts on many of these problems:

\vspace{2mm}

\includegraphics[width=0.8\columnwidth]{../Screen Shot 2021-02-20 at 7.02.54 PM.png} 


\vspace{2mm}

Justification: Collinear is defined that one point lies on the lines that connects the other two points. 

Photo from week 1 of class:

\vspace{2mm}

\includegraphics[width=0.9\columnwidth]{../Screen Shot 2021-02-20 at 9.38.50 PM.png} 

\vspace{4mm}

%%%%%%%%%%%%%%%%%%%%%
\textbf{2) Step: $\triangle FBC \cong \triangle ABD$}

\vspace{2mm}

\includegraphics[width=0.7\columnwidth]{../Screen Shot 2021-02-20 at 7.06.20 PM.png}

\vspace{2mm}

\newpage
Justification: 

\vspace{2mm}

-$\angle FBC$ and $\angle DBC=90 degrees  =>$ interior angles of squares are 90 degrees

\vspace{2mm}

-They both share the same $\angle ABC$, so both $\angle FBC$ and $\angle DBC=90+\angle ABC =>$ both triangles intersecting each other creating same angles.

\vspace{2mm}

-FB=BA and DB=BC $=>$ squares have same lengths on all sides

\vspace{2mm}

-$\triangle FBC \cong \triangle ABD =>$ By SAS 

\vspace{2mm}

Another way to think of this is the property of rotation. 

\vspace{4mm}

%%%%%%%%%%%%%%%%%%%%%
\textbf{3) Area of FBC is half of ABFG and Area of ABD is half of BDLI}

\vspace{2mm}

\includegraphics[width=0.8\columnwidth]{../Screen Shot 2021-02-20 at 7.09.11 PM.png}

\vspace{2mm}

Justification: 

\vspace{2mm}

-$\triangle FBC \cong \triangle ABD$ => Previous step

\vspace{2mm}

-area FBC= $\frac{1}{2}(FB * BA)=\frac{1}{2} ABFG = > $ 4.2 (Area of triangle) and 4.4 (Area of Rhombus)

\vspace{2mm}

-area ABD=$\frac{1}{2}(BD * BI)=\frac{1}{2} BDLI = >$ 4.2 (Area of triangle) and 4.4 (Area of Rhombus)

\vspace{4mm}

%%%%%%%%%%%%%%%%%%%%%
\newpage
\textbf{4) Step: Area of ABFG=area of BDLI...ABFG=BDLI=3.51}

\vspace{2mm}

\includegraphics[width=0.7\columnwidth]{../Screen Shot 2021-02-20 at 7.14.05 PM.png}

\vspace{2mm}

Justification: 

\vspace{2mm}

Because the previous step is true, then

\vspace{2mm}

area $FBC=\frac{1}{2}(3.51)=1.755$ and area ABD=$\frac{1}{2}(3.51)=1.755$.

\vspace{2mm}

Since area of FBC=area of ABD, then it is true that Area of ABFG=area of BDLI. In class I remember we talked about the \textbf{transtivity of equality}; we can use this idea to the justification. 

\vspace{4mm}

%%%%%%%%%%%%%%
\textbf{5) Step: Similiarity}

\includegraphics[width=0.7\columnwidth]{../Screen Shot 2021-02-20 at 7.18.04 PM.png}

\vspace{2mm}

Justification: 

\vspace{2mm}

Same idea with step 2: With $\angle BCK=\angle ACE=$90 degrees $\angle ACB$ and  $\frac{AH}{BD}=\frac{CK}{CE}$ (dilation), $\triangle ACE \sim \triangle BCK$ by 5.5 (SAS similarity theorem).

\vspace{2mm}

%%%%%%%%%%%%%%%%%%%%%%%%%%
\newpage
\textbf{6) Step: Similarity}

\vspace{2mm}

\includegraphics[width=0.7\columnwidth]{../Screen Shot 2021-02-20 at 7.24.23 PM.png}

\vspace{2mm}

Justification: 

\vspace{2mm}

Same as step 3: 

\vspace{2mm}

-area ACE=$\frac{1}{2}$(CE * CL)=$\frac{1}{2}$CILE

\vspace{2mm}

-area BCK=$\frac{1}{2}$(AC*CK)=$\frac{1}{2}$CAHK.

\vspace{2mm}

- area ACE=area BCK

\vspace{2mm}

$\frac{1}{2}(8.13)=\frac{1}{2}(8.13)$

\vspace{2mm}

4.065=4.065

\vspace{2mm}

CILE=CAHK.




\newpage

\textbf{4.9)}Assuming that the ratio of the circumference of a circle of radius 1 to its radius is 2$\pi$, create a convincing argument that the circumference of a circle of radius r is C=2$\pi$r.

Thoughts:

\includegraphics[width=0.7\columnwidth]{../Screen Shot 2021-02-21 at 5.35.01 PM.png}

We are assuming that the circumference of a circle of radius is 1 to its radius of $2\pi$. With Archimedes' Method, we know that circumference of circle with radius 1 is between $\frac{22}{7}$ (aka: $\pi$) and $\frac{223}{71}$. With that idea:

$C=2\pi r$

\vspace{2mm}

Rearranged the formula

\vspace{2mm}

$\pi=\frac{C}{2r}$

\vspace{2mm}

$\frac{1}{\pi}=\frac{2r}{C}$

\vspace{2mm}

Assumption

\vspace{2mm}

$\frac{r}{C}=\frac{1}{2\pi}$

\vspace{2mm}

Cross multiply

\vspace{2mm}

$C=2\pi r$


%\includegraphics[width=0.7\columnwidth]{../Screen Shot 2021-01-26 at 12.29.33 PM.png}



\newpage

\textbf{5.6)} SSS Similarity Theorem: $\triangle$ ABC and $\triangle$ DEF be triangles such that $\frac{AB}{DE}=\frac{BC}{EF}=\frac{CA}{FD}$. Then $\triangle ABC \sim \triangle DEF$.

\vspace{2mm}

Thoughts on paper:

\vspace{2mm}

\includegraphics[width=0.7\columnwidth]{../5.6.png.HEIC.pdf}

\vspace{2mm}


Proof:

\vspace{2mm}

-$\triangle$ ABC and $\triangle$ DEF be triangles $=>$ Given

\vspace{2mm}

-$\frac{AB}{DE}=\frac{BC}{EF}=\frac{CA}{FD} =>$ Given

\vspace{2mm}

-Let P and Q be points on $\overline{DE}$ and $\overline{DF}$  $=>$ Using idea from 5.1. Creates the dilation relationship between $\triangle$ ABC and $\triangle$ DEF.

\vspace{2mm}

-$\overline{DP}=AB =>$ Because of 5.1 and previous step

\vspace{2mm}

-$\angle DPQ=\angle DEF$ and  $\angle DQP=\angle DFE =>$ By corresponding angles theorem (3.4).

\vspace{2mm}

-$\triangle DPQ \sim \triangle DEF =>$ 5.4-AA triangle similarity theorem

\vspace{2mm}

-Since $\overline{DP}=AB$, then $\overline{PQ}=BC$ and $\overline{DQ}=AC$ $=>$ From 5.1

\vspace{2mm}

-$\triangle ABC \sim \triangle DEF$ $=>$ By SSS Similarity Theorem



\newpage

\textbf{6.7)} Power of a Point II Theorem: Let P be a point outside a given circle. Suppose we draw two rays from the point P: one ray intersects the circle at the points A and B (in that order), and the other intersects the circle at the points C and D (in that order). Then PA * PB = PC * PD.

\vspace{2mm}

Thoughts on paper:

\vspace{2mm}

\includegraphics[width=0.8\columnwidth]{../Screen Shot 2021-02-20 at 7.35.38 PM.png}



\vspace{2mm}

-$\angle BAD=\angle DCB$ and $\angle ABC=\angle CDA$ $=>$ By 6.2-Corollary to the Inscribed Angles Theorem.

\vspace{2mm}

-$\triangle PAD \cong \triangle PCB => $ AA congruent theorem 

\vspace{2mm}

-PA * PB = PC * PD $=>$ $\frac{PA}{PC}=\frac{PD}{PB}$


\newpage

\textbf{6.10)} Equidistant Tangents Theorem: Let P be a point outside a given circle. Suppose we draw the two lines from P that are tangent to the circle at points Q and R. Then PQ= PR.

\vspace{2mm}

Thoughts on paper:

\vspace{2mm}

\includegraphics[width=0.8\columnwidth]{../Screen Shot 2021-02-20 at 7.39.31 PM.png}

\vspace{2mm}


-PQ and PR are tangent line $=>$ Given

\vspace{2mm}

-$\angle PQT = \angle PRT =>$ Both have 90 degrees

\vspace{2mm}

-$\overline{RT}=\overline{QT} =>$ Same radius and 6.8-Tangent-Radius Theorem 

\vspace{2mm}

-$\triangle PQT$ and $\triangle PRT$ = 180 degrees $=>$ 3.6-Sum of triangles=180

\vspace{2mm}

-PT=PT $=>$ Reflexive property

\vspace{2mm}

-Then $\triangle PQT \cong \triangle PRT$ $=>$ Reflexive property

\vspace{2mm}

-PQ=PR $=>$ CPCT: 3 pairs of corresponding angles are cong. and 3 pairs of sides are cong. 







\end{document}