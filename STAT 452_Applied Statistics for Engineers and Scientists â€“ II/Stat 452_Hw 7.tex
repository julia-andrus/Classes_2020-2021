 \documentclass{article}
\usepackage{ amsmath, amssymb, soul, color, amsthm}
\usepackage{mathtools}
\usepackage{tikz}
\usetikzlibrary{positioning}
\usepackage{soul, color}
\usepackage{ulem}
\usepackage[linguistics]{forest}

\title{STAT 452\_Hw 7}
\author{Julia Andrus}
\date{}

\begin{document}

\maketitle 

%%%%%%%%%%%%%%%%Chapter 9: Ex 65, Chapter 10: 4, 6, 16, 18, 37, 42

\textbf{9.65)} Obtain a 90\% CI for the ratio of variances by first using the method suggested in the text to obtain a general confidence interval formula.

\vspace{2mm}

Formula: $P(F_{1-\alpha/2, m-1, n-1} \le \frac{S^{2}_{1}/\sigma^{2}_{1}}{S^{2}_{2}/\sigma^{2}_{2}})=1-\alpha$

\vspace{2mm}

We can sub $s_{1} and s_{2}$ out and get 

\vspace{2mm}

$\frac{S^{2}_{2}}{S^{2}_{1}}F_{1-\alpha/2, m-1, n-1} \le \frac{\sigma^{2}_{2}}{\sigma^{2}_{1}} \le \frac{S^{2}_{2}}{S^{2}_{1}}F_{1-\alpha/2, m-1, n-1}$

\vspace{2mm}

Using that formula: 90\%, $\alpha=0.01$, m=n=4

\vspace{2mm}

Now that we have that, we need to use the sample standard deviation: $s=\sqrt{\frac{1}{n-1}}(S_{xx})$

\vspace{2mm}

Putting this into my calculator: for $s_{1}=0.16, s_{2}=0.074$

\vspace{2mm}

Now using the formula: ($\frac{S^{2}_{2}}{S^{2}_{1}}F_{1-\alpha/2, m-1, n-1} \le \frac{\sigma^{2}_{2}}{\sigma^{2}_{1}} \le \frac{S^{2}_{2}}{S^{2}_{1}}F_{1-\alpha/2, m-1, n-1}$)

\vspace{2mm}

$=(\frac{0.074^{2}}{0.16^{2}}(1.08), \frac{0.074^{2}}{0.16^{2}}(9.28)$

\vspace{2mm}

$=(0.23, 1.99)$


\newpage
\textbf{10.4)} Does it appear that true average foam density is not the same for all these manufacturers? Carry out an appropriate test of hypotheses by obtaining as much P-value information as possible, and summarize your analysis in an ANOVA table.

\vspace{2mm}

1: 30.4, 29.2 
2: 27.7, 27.1
3: 27.1, 24.8 
4: 25.5, 28.8

$\overline{x_{1}}=29.8$

$\overline{x_{2}}=27.4$

$\overline{x_{3}}=25.95$

$\overline{x_{4}}=27.15$


\vspace{2mm}

We can easily do this in R:

\vspace{2mm}

\includegraphics[width=.9\columnwidth]{../Screen Shot 2021-02-26 at 2.20.18 PM.png} 






\newpage
\textbf{10.6)} Carry out an analysis of variance F test at significance level .01, and summarize the results in an ANOVA table.

\vspace{2mm}

Shown in R:

\vspace{2mm}

\includegraphics[width=.9\columnwidth]{../Screen Shot 2021-02-26 at 2.39.21 PM.png} 

\hl{Answer:} Since our p-value is smaller than 0.01, we can reject $H_{0}$.




\newpage
\textbf{10.16a)} Is it plausible that the variances of the five axial stiff- ness index distributions are identical?

\vspace{2mm}

At level 10 we have the sd=44.51 which is the greatest sd. At level 8 we have sd=20.83, which is the lowest sd. Since 44.51 $>$ 20.83 by twice as much, we can say that the variances are equal.

\vspace{5mm}


\textbf{10.16b)} Use the output (without reference to our F table) to test the relevant hypotheses.

\vspace{2mm}

$H_{0}=\mu_{i}=\mu_{j}$

\vspace{2mm}

$H_{a}$:  at least two $\mu$'s are different

\vspace{2mm}

F=10.48, then the p-value is under the F curve with: $P(F>f)=0$. Looking at the back of the book (Table A.10) we get that our df=4 and 30. With that value, we can reject $H_{0}$ because there is difference in the axial stiffness for different plate lengths.

\vspace{5mm}



\textbf{10.16c)} Use the Tukey intervals given in the output to determine which means differ, and construct the corresponding underscoring pattern.

\vspace{2mm}

I=5, J=7, $\alpha=0.05$, 

\vspace{2mm}

$w=Q_{\alpha, I, I(J-1)} (\sqrt{\frac{MSE}{J}})$

\vspace{2mm}

$Q_{\alpha, I, I(J-1)}=Q_{0.05, 5, 5(7-1)} =Q_{0.05, 5, 30}=4.1$

\vspace{2mm}

$(\sqrt{\frac{MSE}{J}})=(\sqrt{\frac{1049}{7}})=12.2416$

\vspace{2mm}

$w=Q_{\alpha, I, I(J-1)} (\sqrt{\frac{MSE}{J}})=(4.1)(12.2416)=50.1906$

\vspace{2mm}

Since the difference is given to us, we can draw out the pattern

\vspace{2mm}

\includegraphics[width=.9\columnwidth]{../Screen Shot 2021-02-26 at 11.02.40 PM.png}


\textbf{10.18a)} Perform an F test at level $\alpha=$.05.

\vspace{2mm}

1: 13, 17, 7, 14

\vspace{2mm}

2: 21, 13, 20, 17

\vspace{2mm}

3: 18, 15, 20, 17

\vspace{2mm}

4: 7, 11, 18, 10

\vspace{2mm}

5: 6, 11, 15, 8

\vspace{2mm}

I=5, J=4, n=20

\vspace{2mm}

$\overline{x_{1}}=12.75$
$x_{1}=51$

\vspace{2mm}


$\overline{x_{2}}=17.75$
$x_{2}=71$

\vspace{2mm}

$\overline{x_{3}}=17.5$
$x_{3}=70$

\vspace{2mm}

$\overline{x_{4}}=11.5$
$x_{4}=46$

\vspace{2mm}

$x_{..}=238$

\vspace{2mm}

$F=\frac{MSTr}{MSE}$

\vspace{2mm}

Formula: SST=SStr+SSE

\vspace{2mm}

SST=$\sum^{I}_{i=1} \sum^{J}_{j} x^{2}_{ij} -\frac{1}{IJ}x^{2}$

$=((13)^{2}+(17)^{2}+.....+(8)^{2}-\frac{(238)^{2}}{20}=4280-2832.2=1447.8$

\vspace{2mm}

$SStr=\frac{1}{J}(\sum x_{i}^{2}-\frac{1}{IJ}x^{2}:$

$=\frac{1}{4}(51)^{2}+(71)^{2}+(70)^{2}+(46)^{2}-\frac{(238)^{2}}{20}=\frac{1}{4}(14658)-2832.2$
$=3664.5-2832.2=832.3$

\vspace{2mm}

Formula: SST=SStr+SSE

\vspace{2mm}

SSE=$1447.8-832.3=615.5$


$MSTr=\frac{SStr}{I-1}=\frac{832.3}{5-1}=\frac{832.3}{4}=208.06$

\vspace{2mm}

$MSE=\frac{SSE}{I(J-1)}=\frac{615.5}{5(4-1)}=\frac{615.5}{15}=41.03333$

\vspace{2mm}

Decided do use R, since it is a shortcut:

\vspace{2mm}

\includegraphics[width=.9\columnwidth]{../Screen Shot 2021-02-26 at 3.38.22 PM.png}

\vspace{2mm}

\hl{Answer:} I have done everything in R. Since our p-value is 0.03 and $\alpha=0.05$, then we can reject $H_{0}$ because p-value $< \alpha$


\vspace{5mm}

\textbf{10.18b)} What happens when Tukey’s procedure is applied?

\vspace{2mm}

Using the equation: $w=Q_{\alpha, I, I(J-1)} (\sqrt{\frac{MSE}{J}})$

\vspace{2mm}

I=5, J=4, n-1=20-1=19

\vspace{2mm}

$Q_{\alpha, I, I(J-1)}=Q_{0.05, 5, 15}=>$ Table A.10 =4.37

\vspace{2mm}

$(\sqrt{\frac{MSE}{J}})=(\sqrt{\frac{14.37}{4}})=1.89539$

\vspace{2mm}

$w=(4.37)(1.89539)=8.28285$

\vspace{2mm}

\hl{Answer:} Finding the differences of mean in R, we can say there is no significance differences since all of results are less than w=8.28285:

\vspace{2mm}


\includegraphics[width=.9\columnwidth]{../Screen Shot 2021-02-26 at 4.06.21 PM.png}




\vspace{2mm}




\newpage
\textbf{10.37)} State and test the relevant hypotheses at significance level .05, and then carry out a multiple comparisons analysis if appropriate.

\vspace{2mm}

\includegraphics[width=1.0\columnwidth]{../Screen Shot 2021-02-26 at 6.02.33 PM.png}

\vspace{2mm}

Since our p-value is 0.000187 $<$ 0.05=$\alpha$, then we can reject $H_{0}$ concluding that there is sufficient evidence to support the claim that population means are different.

\vspace{2mm}

\vspace{2mm}

For the Multiple comparisons analysis:

\vspace{2mm}

$T=Q_{\alpha, I, I(J-1)} (\sqrt{\frac{MSE}{J}})$

\vspace{2mm}

I=5, J=6, n=30, n-I=25, $\alpha=0.05$

\vspace{2mm}

$Q_{\alpha, I, I(J-1)}=Q_{0.05, 5, 5(6-1)}=Q_{0.05, 5, 25} =>$ Table A.10 $=4.15$

\vspace{2mm}

MSE=0.914

Now with the formula $w=Q_{\alpha, I, I(J-1)} (\sqrt{\frac{MSE}{J}})$

\vspace{2mm}

$=Q_{\alpha, I, I(J-1)} (\sqrt{\frac{MSE}{J}})=4.15(\sqrt{\frac{4.15}{6}})=1.61974$

\vspace{2mm}

For the 95\% confidence simultaneous region: 

\vspace{2mm}

$(\mu_{2}-\mu_{1}, \mu_{3}-\mu_{1}, \mu_{4}-\mu_{1}, \mu_{5}-\mu_{1},$

\vspace{2mm}

$\mu_{3}-\mu_{2}, \mu_{4}-\mu_{2}, \mu_{5}-\mu_{2},$

\vspace{2mm}

$\mu_{4}-\mu_{3}, \mu_{5}-\mu_{3},$

\vspace{2mm}

$\mu_{5}-\mu_{4})$

\vspace{2mm}

=

\vspace{2mm}

$[(15.95-13.68)-1.6194, (15.95-13.68)+1.6194$

(13.67-13.68)-1.6194, (13.67-13.68)+1.6194


.

.

.

.

.

$]$

=

\vspace{2mm}

$[(0.6506, 3.8894)$

$(-1.6294,1.6094)$

.

.

.

.

.

$]$

\vspace{2mm}

Now finding the differences of the means:

\vspace{2mm}

$\overline{x_{2}}-\overline{x_{1}}$

$\overline{x_{3}}-\overline{x_{1}}$

$\overline{x_{4}}-\overline{x_{1}}$

.

.

$\overline{x_{3}}-\overline{x_{2}}$

.

.


$\overline{x_{4}}-\overline{x_{3}}$

.

.

\vspace{2mm}

If the results of the differences of the means is $<$ then w=1.6194, then that the means the means belong in the same group.

\vspace{2mm}

Did this in R:

\vspace{2mm}

\includegraphics[width=1.0\columnwidth]{../Screen Shot 2021-02-26 at 7.17.59 PM.png}



\newpage
\textbf{10.42a)} State and test the relevant hypotheses at significance level .05. 

\vspace{2mm}


$H_{0}: \mu_{1}=\mu_{2}=\mu_{3}$

\vspace{2mm}

$H_{a}$: at least two of the $\mu$'s are different.


\vspace{2mm}

We are hinted that: $\sum \sum x^{2}_{ij}=13,659.67$ and CF=13,598.36

\vspace{2mm}

So 

\vspace{2mm}

SST=13,659.67-13,598.36=61.31

\vspace{2mm}

$SSTr=\frac{(204.7)^{2}}{8}+\frac{(134.6)^{2}}{5}+\frac{(169.0)^{2}}{6}-13598=13621.4-13598=22.999=23$

\vspace{2mm}

SSE=61.31-23=38.31

\vspace{2mm}

$F_{.05,I-1,n-I}=F_{0.05, 3-1, 19-1}=F_{0.05, 2, 18} =>$ Table A.10 $=>2.97$ 

\vspace{2mm}

f=$\frac{MSTr}{MSE}$

\vspace{2mm}

$MSTr=\frac{J}{I-1}[(\overline{X}_{1}-\overline{X}_{...})^{2}+(\overline{X}_{2}-\overline{X}_{..})^{2}+....\overline{X}_{I}-\overline{X}_{..})^{2}]$

\vspace{2mm}

But since we already have MSTr, then $MSTr=\frac{SStr}{I-1}=\frac{23}{3-1}=11.5$

\vspace{2mm}

$MSE=\frac{SSE}{n-I}=\frac{38.31}{22-3}=2.01632$

\vspace{2mm}


$f=\frac{MSTr}{MSE}=\frac{11.5}{2.01632}=5.70346$

\vspace{2mm}

\hl{Answer:} Since $F_{0.05, 2, 18}=2.97 < f=5.70346$, then $H_{0}$ is rejected.



\vspace{5mm}
 
 
 
\textbf{10.42b)} Investigate differences between iris colors with respect to mean cff.

\vspace{2mm}

To invesitgate the differences we have to find w:

\vspace{2mm}

$w=Q_{\alpha, I, n-1} (\sqrt{\frac{MSE}{2}(\frac{1}{J_{i}}+\frac{1}{J_{j}}})$

\vspace{2mm}

Finding sample sizes closest to each other for $J_{1},J_{2}...J_{I}$. Sample sizes for green and blue 
are closer together therefore $J_{i}=5, J_{j}=6$

\vspace{2mm}

$=Q_{0.05, 3, 19-1} (\sqrt{\frac{2.01632}{2}(\frac{1}{5}+\frac{1}{6}})$

\vspace{2mm}

$=Q_{0.05, 3, 18}(\sqrt{.369659})$

\vspace{2mm}

Table A.10

\vspace{2mm}

$=3.61(.607996)$

\vspace{2mm}

$w=2.19$

\vspace{2mm}

Now for the difference of means:

\vspace{2mm}

$\overline{x}_{2}-\overline{x}_{1}=26.92-25.59=1.33<2.19$

\vspace{2mm}

$\overline{x}_{3}-\overline{x}_{1}=28.17-25.59=2.58>2.19$

\vspace{2mm}

$\overline{x}_{3}-\overline{x}_{2}=28.17-26.92=1.25<2.19$

\vspace{2mm}

\hl{Answer:} As you can see, $\mu_{3}$ (Blue) and $\mu_{1}$(Green) appear to be significantly different.












\end{document}