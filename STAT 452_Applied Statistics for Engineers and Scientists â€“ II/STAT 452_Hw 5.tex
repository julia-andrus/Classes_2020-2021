\documentclass{article}
\usepackage{amsmath, amssymb}
\usepackage{ graphicx, tikz}
\usetikzlibrary{calc}
\usepackage{soul, color}
\usepackage{ulem} %stricking text
\title{Hw\_5} 
\author{Julia Andrus}
\date{}
 
 \begin{document}
\maketitle

%%%%%%%%%%%%%%%%%%%%%%%%%%  Ex 8.48(a and b), 8.74, 9.6, 9.12, 9.28
\textbf{48a)} 51 homes found and 41/51 had problems. Suppose 51 were randomly sampled from the population of all homes having Chinese drywall. Does the data provide strong evidence for concluding that more than 50\% of all homes with Chinese drywall have electrical/environmental problems? Carry out a test of hypotheses using a $\alpha= .01$.

\vspace{2mm}

$H_{0}: \mu=50\%=0.50$

\vspace{2mm}

$H_{a}: \mu>50\%=0.50$

\vspace{2mm}

We need to solve for the sample proportion: $\hat{p}=\frac{x}{n}$

$$\hat{p}=\frac{41}{51}$$
$$=0.804$$

Now using the test statistics: $Z=\frac{\hat{p}-p}{\sqrt{\frac{p(1-p)}{n}}}$

\vspace{2mm}

p=0.50

\vspace{2mm}

$\hat{p}=0.804$

\vspace{2mm}

$$=\frac{0.804-0.50}{\sqrt{\frac{0.50(1-0.50)}{51}}}$$
$$\frac{0.304}{0.070}$$
$$Z=4.34$$

Now we input 3.49 instead of 4.34 because 3.34 is not in the table:

\vspace{2mm}

$1-\phi(4.34)<1-\phi(3.49)=0.0003$

\vspace{2mm}

\hl{Answer:} Since $0.0003 < .01$, then $H_{0}$ can be rejected and thus this does give strong evidence for concluding that more than 50\% of all homes with Chinese drywall have electrical/environmental problems.



%%%%%%%%%%%%%%
\vspace{5mm}

\textbf{48b)} Calculate a lower confidence bound using a confidence level of 99\% for the percentage of all such homes that have electrical/environmental problems.

The lower confidence bound formula: $\hat{p}-Z_{\frac{\alpha}{2}}\sqrt{\frac{\hat{p}(1-\hat{p})}{n}}$

\vspace{2mm}

To find $Z_{\frac{\alpha}{2}}$: 99\% CI

\vspace{2mm}

1-0.99=0.01

\vspace{2mm}

$\frac{0.01}{2}=0.005$

\vspace{2mm}

Looking at the table we get 2.33.

\vspace{2mm}

$$=0.804-2.33\sqrt{\frac{0.804(1-0.804)}{51}}$$
$$=0.804-2.33(0.056)$$
$$=0.804-0.13048$$
$$=0.67$$

\vspace{2mm}

\hl{Answer:} 0.67



\newpage
%%%%%%%%%%%%%%%%%%%%%%%%%% 8.74, 9.6, 9.12, 9.28
\textbf{8.74)}


\newpage
%%%%%%%%%%%%%%%%%%%%%%%%%% 9.6, 9.12, 9.28
\textbf{9.6a)} Assuming that $\sigma_{1}=1.6$ and $\sigma_{2}=1.4$, test $H_{0}: \mu_{1}-\mu_{2}=0$ and $H_{a}: \mu_{1}-\mu_{2}$ at level .01.

Given: 

$\overline{X}=18.12, m=40, \sigma_{1}=1.6$ 


$\overline{Y}=16.87, n=32, \sigma_{1}=1.4, \sigma_{2}=1.4$ 

$H_{0}: \mu_{1}-\mu_{2}=0$ and $H_{a}: \mu_{1}-\mu_{2}$ 

Since sigma is known, use z-test:

$$Z=\frac{\overline{X}-\overline{Y}-(\mu_{1}-\mu_{2})}{\sqrt{\frac{\sigma^{2}_{1}}{m}+\frac{\sigma^{2}_{2}}{n}}}$$

 $\mu_{1}-\mu_{2}=0$

$$=\frac{18.12-16.87-0}{\sqrt{\frac{1.6^{2}}{40}+\frac{1.4^{2}}{32}}}$$
$$=\frac{1.25}{\sqrt{0.064+0.06125}}$$
$$=\frac{1.25}{0.289}$$
$$=3.53$$

P-value:
 $1-\theta(3.53)=1-0.999742$ (I used R)
 
 $=0.0002158$
 
 Since p-value=$0.0002158<0.01=\alpha$, then $H_{0}$ is rejected at significance level of 0.01.
 
 
 
 %%%%%%%%%%%%%%%%%%%%%
 \vspace{3mm}
 \textbf{9.6b)} 


\newpage
%%%%%%%%%%%%%%%%%%%%%%%%%%  48(a and b)
\textbf{8.74)}

\textbf{9.25a)}


\newpage
\textbf{9.25a)}

\textbf{9.25b)} Construct boxplot and compare.

In $\le 89$ box plot, it shows that the data is right skewed. In $ \ge 93$ boxplot, it shows that the data is left skewed. The one wine with rating 93 or higher costs more than the wine with ratings of 89. 


\end{document}
