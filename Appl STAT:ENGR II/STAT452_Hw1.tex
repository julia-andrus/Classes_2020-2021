\documentclass{article}
\usepackage{amsmath, amssymb}
\usepackage{soul, color}
\usepackage{ulem} %stricking text
\title{Hw 1-Ex: 3, 8, 11, 12, 22, 32, 36}
\author{Julia Andrus}
\date{}

\begin{document}

\maketitle
%%%%%%%%%%%%%%%%%%%%%%%%%%%%%%%%%%%%%%
\textbf{3a)} Calculate a point estimate of the mean value of coating thickness, and state which estimator you used.

\vspace{2mm}
Sample of observations: 

\vspace{5mm}

$\begin{matrix}
0.83 & 0.88 & 0.88 & 1.04 & 1.09 &1.12 & 1.29 & 1.31 \\

1.48 & 1.49 & 1.59 & 1.62 & 1.65 & 1.71 & 1.76 & 1.83
\end{matrix}$

\vspace{5mm}

Using Sample mean to find the point estimate of mean value: 

 
$$\bar{x} =\frac{\sum_{ i=1}^{n} x_{i}}n =\frac{1}n \frac{\sum_{ i=1}^{n} x_{i}}n$$

 
 So,
 
 
 
 $x_{1}+x_{2}+....x_{16}=0.83+0.88+0.88+...1.83$=21.57
 
 
 n=16
 \vspace{5mm}
 
 
 \hl{Answer:} $ \frac{1}n \frac{\sum_{ i=1}^{n} x_{i}}n=\frac{1}{16} (21.57)=1.3481$
 
 \vspace{2mm}

  
  %%%%%%%%%%%%%%%%%%%%%%%%%%%%%%%%%%%%%%
 \textbf{3b)} Calculate a point estimate of the median of the coat­ ing thickness distribution, and state which estimator you used. 
 
 \vspace{5mm}
 
 Since there are 16 ordered observations, 1.31 and 1.48 are our two values. By using the Sample Median, we can  get our point estimate of the median:
 
  \vspace{5mm}

\hl{Answer:} $\bar{x}=\frac{1.31+1.48} 2=1.395$
\vspace{5mm}
%%%%%%%%%%%%%%%%%%%%%%%%%%%%%%%%%%%%%%

\textbf{3c)} Calculate a point estimate of the value that separates the
largest 10\% of all values in the thickness distribution from the remaining 90\%, and state which estimator you used. (Hint: Express what you are trying to estimate in terms of $\mu$ and $\sigma$.)
  \vspace{5mm}
  


Finding Sample Variance:

$$s^2=\frac{1}{n-1}\frac{\sum(x_{i}-\bar{x})^2}{n-1}= \frac{1}{n-1}\sum x_{i}^{2} -\frac{1}{n}(\sum x_{i})^2$$

\newpage

$\begin{matrix}
0.83 & 0.88 & 0.88 & 1.04 & 1.09 &1.12 & 1.29 & 1.31 \\

1.48 & 1.49 & 1.59 & 1.62 & 1.65 & 1.71 & 1.76 & 1.83

\end{matrix}$

  \vspace{3mm}

\textbf{$x_{i}^{2}$:}
  \vspace{2mm}

$\begin{matrix}
0.6889 & 0.7744 & 0.7744 & 1.0816 & 1.1881 & 1.2544 & 1.6641 & 1.7161 \\

2.1904 & 2.2201 & 2.5281 & 2.6244 & 2.7225 & 2.9241 & 3.0976 & 3.3489

\end{matrix}$
\vspace{3mm}

Now we can find $s^2$:
\vspace{3mm}

$s^2=\frac{1}{n-1}\frac{\sum(x_{i}-\bar{x})^2}{n-1}= \frac{1}{n-1}\sum x_{i}^{2} -\frac{1}{n}(\sum x_{i})^2$

$s^2=\frac{1}{n-1}(0.6889+0.7744+...3.3489-\frac{1}{16} (0.83+0.88+...1.83)^2) $

$s^2=\frac{1}{n-1}(30.7981-\frac{1}{16}(21.57)^2)$

$s^2=\frac{1}{n-1}(1.719)$

$s^2=\frac{1}{16-1}(1.719)=0.1146$

\vspace{3mm}
Finding the sample Standard Deviation: $s=\sqrt{s^2}=\sqrt{0.1146}=0.3385$

\vspace{2mm}

We now have to use $\mu +z_{1}-_{\alpha}\sigma$ to find the 90th percentile:  

\vspace{2mm}
  
  $\mu=\bar{x}=1.3481$
  $\alpha=0.1 (10\%)$
  $z_{1}-_{\alpha}=1.28$
  $\sigma=s=0.3385$
  
  $\mu +z_{1}-_{\alpha}\sigma=1.3481+1.28(0.3385)=1.7814$
  
  \vspace{3mm}
  
\hl{Answer:} the 90th percentile is 1.7814


\vspace{5mm}
%%%%%%%%%%%%%%%%%%%%%%%%%%%%%%%%%%%%%%
\textbf{3d)}  Estimate P(X , 1.5), i.e., the proportion of all thick­ ness values less than 1.5.

\begin{equation}
P(X<1.5) =P(\frac{X-\bar{x}}s <\frac{1.5-1.3481}{0.3385})=P(Z< 0.45)
\end{equation}

We can use the appendix or use R to find our answer: 
\vspace{2mm}

\hl{Answer:} $P(Z< 0.45)=0.6737$
\vspace{5mm}
%%%%%%%%%%%%%%%%%%%%%%%%%%%%%%%%%%%%%%

\textbf{3e)}  What is the estimated standard error of the estimator that you used in part (b)? 
\vspace{2mm}

The estimated standard error of the estimator:
\vspace{3mm}

\hl{Answer:} $\bar{X}=\sqrt{\frac{\sigma^2}{n}}=\frac{\sigma}{\sqrt{n}}=\frac{s}{\sqrt{n}}=\frac{0.3385}{\sqrt{16}}=0.0846$

%%%%%%%%%%%%%%%%%%%%%%%%%%%%%%%%%%%%%%
\newpage

\textbf{8a)} In a random sample of 80 components of a certain type, 12 are found to be defective. Give a point estimate of the proportion of all such components that are not defective. 

Since there are 12 that are defective from the 80 components, 68 are not defective (80-12). So the point estimate of components that are not defective is:

\hl{Answer:} $\hat{p}=\frac{68}{80}=0.85$

\vspace{5mm}

%%%%%%%%%%%%%%%%%%%%%%%%%%%%%%%%%%%%%%
\textbf{8b)} A system is to be constructed by randomly selecting two of these components and connecting them in series, as shown here.The series connection implies that the system will func­tion if and only if neither component is defective (i.e., both components work properly). Estimate the propor­tion of all such systems that work properly. [Hint: If p denotes the probability that a component works properly, how can P(system works) be expressed in terms of p?]

\vspace{3mm}

Since both components work properly, then P(system works)=$p^2$

So, $\hat{p^2}=(0.85)^2=0.723$
\vspace{2mm}

\hl{Answer:}  $\hat{p}^{2}=(0.85)^2=0.723$

%%%%%%%%%%%%%%%%%%%%%%%%%%%%%%%%%%%%%%
\newpage

\textbf{11a)}  Show that $(X_{1}/n_{1})-(X_{2}/n_{2})$ is an unbiased estimator for $p_{1} - p_{2}$. [Hint: $E(X_{i} ) = n_{i} p_{i}$ for i=1,2.]
\vspace{3mm}

Since we are given the hint of $E(X_{i} ) = n_{i} p_{i}$ for i=1,2, then,

$$E(X_{1} ) = n_{1} p_{1}$$
$$E(X_{2} ) = n_{2}p_{2}$$

So,

$$E(\frac{X_{1}}{n_{1}})-E(\frac{X_{2}}{n_{2}} )$$


$$\frac{n_{1} p_{1}}{n_{1}}-\frac{n_{2}p_{2}}{n_{2}}$$

$$p_{1} - p_{2}$$


Since $E(\frac{X_{1}}{n_{1}})-E(\frac{X_{2}}{n_{2}} )=p_{1}-p_{2}$, then $\frac{X_{1}}{n_{1}})-\frac{X_{2}}{n_{2}}$ is an unbiased estimate of $p_{1}-p_{2}$.
%%%%%%%%%%%%%%%%%%%%%%%%%%%%%%%%%%%%%%
\vspace{3mm}

\textbf{11b)} What is the standard error of the estimator in part (a)?
\vspace{3mm}

We need to use the variance of the binomial distribution:

$$V(X_{1})=\sigma_{1}^{2}=n_{1}p_{1}q_{1}=n_{1}p_{1}(1-p_{1})$$
$$V(X_{2})=\sigma_{2}^{2}=n_{2}p_{2}q_{2}=n_{2}p_{2}(1-p_{2})$$

Finding the variance of $\frac{X_{1}}{n_{1}}-\frac{X_{2}}{n_{2}}$:
\vspace{3mm}

$V(\frac{X_{1}}{n_{1}}-\frac{X_{2}}{n_{2}})=\frac{1}{n_{1}^{2}}V(X_{1})+\frac{1}{n_{2}^{2}}V(X_{2})$
 $=\frac{1}{n_{1}^{2}}n_{1}p_{1}(1-p_{1})+\frac{1}{n_{2}^{2}}n_{2}p_{2}(1-p_{2})=\frac{p_{1}(1-p_{1})}{n_{1}}+\frac{p_{2}(1-p_{2})}{n_{2}}$
 
 \vspace{3mm}
 
\hl{Answer:}  Standard error of the estimator:

$$(\frac{X_{1}}{n_{1}}-\frac{X_{2}}{n_{2}})=\sqrt{V(\frac{X_{1}}{n_{1}}-\frac{X_{2}}{n_{2}})}=\sqrt{\frac{p_{1}(1-p_{1})}{n_{1}}+\frac{p_{2}(1-p_{2})}{n_{2}}}$$


\vspace{3mm}
%%%%%%%%%%%%%%%%%%%%%%%%%%%%%%%%%%%%%%
\textbf{11c)} How would you use the observed values $x_1$ and $x_2$ to estimate the standard error of your estimator?

We can let $\hat{p_{1}}=\frac{x_{1}}{n_{1}}$ and $\hat{p_{2}}=\frac{x_{2}}{n_{2}}$, then the standard estimator is
\vspace{2mm}

\hl{Answer:} $$(\frac{X_{1}}{n_{1}}-\frac{X_{2}}{n_{2}})=\sqrt{V(\frac{X_{1}}{n_{1}}-\frac{X_{2}}{n_{2}})}=\sqrt{\frac{p_{1}(1-p_{1})}{n_{1}}+\frac{p_{2}(1-p_{2})}{n_{2}}}=\sqrt{\frac{\frac{x_{1}}{n_{1}}(1-\frac{x_{1}}{n_{1}})}{n_{1}}+{\frac{\frac{x_{2}}{n_{2}}(1-\frac{x_{2}}{n_{2}})}{n_{2}}}}$$
\newpage

%%%%%%%%%%%%%%%%%%%%%%%%%%%%%%%%%%%%%%
\textbf{11d)} If $n_1=n_2=200$, $x_{1}=127$, and $x_{2}=176$, use the estimator of part (a) to obtain an estimate of 
$p_{1}-p_{2}$. 

$n_1=200$

$n_2=200$

$x_{1}=127$

$x_{2}=176$
\vspace{3mm}

Since $(\frac{X_{1}}{n_{1}})-(\frac{X_{2}}{n_{2}} )$ is an estimator of $p_{1} - p_{2}$, then:

$$\hat{p_{1}}-\hat{p_{2}}=(\frac{X_{1}}{n_{1}})-(\frac{X_{2}}{n_{2}} )=\frac{127}{200}-\frac{176}{200}=-0.245$$

\hl{Answers:} $\hat{p_{1}}-\hat{p_{2}}=-0.245$

%%%%%%%%%%%%%%%%%%%%%%%%%%%%%%%%%%%%%%
\vspace{3mm}
\textbf{11e)} Use the result of part (c) and the data of part (d) to estimate the standard error of the estimator.
\vspace{3mm}

Using the result from c:

$(\frac{X_{1}}{n_{1}}-\frac{X_{2}}{n_{2}})=\sqrt{V(\frac{X_{1}}{n_{1}}-\frac{X_{2}}{n_{2}})}=\sqrt{\frac{p_{1}(1-p_{1})}{n_{1}}+\frac{p_{2}(1-p_{2})}{n_{2}}}=\sqrt{\frac{\frac{x_{1}}{n_{1}}(1-\frac{x_{1}}{n_{1}})}{n_{1}}+{\frac{\frac{x_{2}}{n_{2}}(1-\frac{x_{2}}{n_{2}})}{n_{2}}}}$

$$=(\frac{X_{1}}{n_{1}}-\frac{X_{2}}{n_{2}})=\sqrt{\frac{\frac{127}{200}(1-\frac{127}{200})}{200}+{\frac{\frac{176}{200}(1-\frac{176}{200})}{200}}}$$

$$=0.0411$$


\hl{Answer:} Standard error=0.0411


  %%%%%%%%%%%%%%%%%%%%%%%%%%%%%%%%%%%%%%

\vspace{3mm}
\newpage

\textbf{12)} Certain type of fertilizer has an expected yield per arce of $\mu_{1}$with variance $\sigma^{2}$, where as the expected yield for a second type of fertilizer is $\mu_{2}$ with the same variance $\sigma^{2}$. Sample variances pf yields based on sample sizes $n_{1}, n_{2}$ of the two fertilizers: $S_{1}^{2}, S_{2}^{2}$. Show the estimator is unbiased of $\sigma^{2}$:

$$\hat{\sigma}^{2}=\frac{(n_{1} -1)S_{1}^{2}+(n_{2} - 1)S_{2}^{2}}{n_{1}+n_{2} -2}$$

Now solving for unbiased estimator:
$$E(\frac{(n_{1}-)S_{1}^{2} + (n_{2} - 1)S_{2}^{2}}{n_{1} + n_{2} -2})$$

Multiply:

$$=\frac{n_{1}-1} {n_{1} + n_{2} -2}E(S_{1})^{2} +\frac{n_{2}-1} {n_{1} + n_{2} -2}E(S_{2})^{2}$$

To help us understand what $E(S_{1})^{2} \& E(S_{2})^{2}$, we have to use a long proof from the book (Chp. 6.1; pg 253):

$$E(S^{2})=\frac{1}{n-1}[\sum X_{i}^{2}-\frac{(\sum X_{i})^{2}}{n}]$$
$$E(S^{2})=\frac{1}{n-1}[\sum E(X_{i}^{2})-\frac{1}{n}E(\sum X_{i})^{2}]$$
$$ . $$
$$ . $$
$$ . $$
$$=\sigma^{2}$$

With this proof, we can now start solving: 

$$\hat{\sigma}^{2}=\frac{(n_{1} -1)S_{1}^{2}+(n_{2} - 1)S_{2}^{2}}{n_{1}+n_{2} -2}$$
$$=\frac{n_{1}-1} {n_{1} + n_{2} -2}\sigma^{2} +\frac{n_{2}-1} {n_{1} + n_{2} -2}\sigma^{2}$$
Combine:
$$=\frac{n_{1}+n_{2}-2} {n_{1} + n_{2} -2}\sigma^{2}$$
Cancel:

$$=\sigma^{2}$$


\hl{Answer:}
So the estimators $S_{1}^{2}, S_{2}^{2}$ are unbiased estimators of the variance.




%%%%%%%%%%%%%%%%%%%%%%%%%%%%%%%%%%%%%%
\newpage

\textbf{22a)}  Use the method of moments to obtain an estimator of $\theta$, and then compute the estimate for this data.

\vspace{2mm}
 $$f(x ; \theta)=
\begin{cases}
 (\theta+1)x^\theta & 0\le x \le \theta \\
 0 & \text{otherwise} 
\end{cases}$$ 

Doing the integral:

$$\bar{X}=\int_{0}^{1}x(\theta+1)x^{\theta}dx=(\theta+1)*\frac{x^\theta + 2}{\theta +2}|_{0}^{1}$$
$$\hspace{.65in}=\frac{\theta +1}{\theta + 2}$$

Now we can find $\hat{\theta}$:
\vspace{2mm}

$\bar{X}=\frac{\theta +1}{\theta + 2}$

$\bar{X}=\frac{\hat{\theta} +1 + (1 - 1)}{\hat{\theta} + 2}$

$\bar{X}=\frac{\hat{\theta} +2}{\hat{\theta} + 2}-\frac{1}{\hat{\theta} + 2}$

$\bar{X}-1=-\frac{1}{\hat{\theta} +2}$

$\hat{\theta}+2=-\frac{1}{\bar{X}-1}$

$\hat{\theta}=-\frac{1}{1-\bar{X}}-2$
\vspace{2mm}

Now we have to find $\bar{X}$:

$$\bar{X}=\frac{X_{1}+X_{2}+...X_{10}}{10}$$
$$\hspace{.5in}=\frac{0.92+0.79+...+0.88}{10}$$
$$=0.8$$

Finally, finding the estimate $\hat{\theta}$:

\vspace{2mm}
$\hat{\theta}=-\frac{1}{1-\bar{X}}-2$

$\hspace{.1in}=-\frac{1}{1-0.8}-2$

\hl{Answer:} $\hat{\theta}=3$
\newpage

\vspace{3mm}
%%%%%%%%%%%%%%%%%%%%%%%%%%%%%%%%%%%%%%
\textbf{22b)} Obtain the maximum likelihood estimator of $\theta$, and then compute the estimate for the given data. 

The likelihood function for this would be:

$$f(x_{1}, x_{2},....x_{n}; \theta)= (\theta + 1)x_{1}^{\theta} * (\theta + 1)x_{2}^{\theta}* (\theta + 1)x_{n}^{\theta}$$
$$\hspace{.6in}=(\theta +1)^{2} * (x_{1}, x_{2},....x_{n})^{\theta}$$

To get the maximum likelihood estimator we have to include the log likelihood function:

$$\ln f(x_{1}, x_{2},... x_{n}; \theta)= \ln[(\theta+1)^{n} * (x_{1} *x_{2}*...x_{n})^{\theta}]$$

$$\hspace{1.25 in}=n*\ln(\theta+1)+\theta * \sum_{i=1}^{n}  \ln  x_{i}$$

If we took the derivative of the log likelihood function respect to $\theta$and equating it to 0 which the maximum likelihood estimator is obtained:

$$\frac{d}{d\theta}f(x_{1}, x_{2},... x_{n}; \theta)=\frac{d}{d\theta}[2 * \ln(\theta+1) + \theta* \sum_{i=1}^{n}  \ln  x_{i}]$$
$$\hspace{.7in}=n*\frac{1}{(\theta+1)}+\sum_{i=1}^{n}  \ln  x_{i}$$

Now solve for $\hat{\theta}$: 

$$n*\frac{1}{(\hat{\theta}+1)}+\sum_{i=1}^{n}  \ln  x_{i}=0$$
$$n*\frac{1}{(\hat{\theta}+1)}=-\sum_{i=1}^{n}  \ln  x_{i}$$
$$\frac{1}{(\hat{\theta}+1)}=\frac{-\sum_{i=1}^{n}  \ln  x_{i}}{n}$$
$$\hat{\theta}+1=\frac{n}{-\sum_{i=1}^{n}  \ln  x_{i}}$$
$$\hat{\theta}=-\frac{n}{\sum_{i=1}^{n}  \ln  x_{i}}-1$$
\vspace{2mm}

Now with $x_{1}, x_{2}.....,x_{10}$ and using the calculator, then

$$\hat{\theta}=-\frac{10}{-2.4295} -1$$


$$\hspace{.11in}=3.12$$

\begin{center}
\hl{Answer:} $\hat{\theta}=3.12$
\end{center}


%%%%%%%%%%%%%%%%%%%%%%%%%%%%%%%%%%%%%%
\newpage

\textbf{32a)} Let $X_{1},...X_{n}$ be a random sample from a uniform distribution on $[0,\theta]$. Then the mle of       $\theta$ is $\hat{\theta}=Y=\max(X_{i})$. Use the fact that $Y \le y$ iff each $X_{i} \le y$ to the derivative the cdf of Y. Then show that the pdf of $Y=\max(X_{i})$ is: 

 $$f_{y}(y)=
\begin{cases}
 \frac{ny^{n-1}}{\theta^{n}} & 0\le y \le \theta \\
 0 & \text{otherwise} 
\end{cases}$$ 

cdf:

$$F_{Y}(y)=P(Y \le y)=P(max(X_{i}) \le y$$

all the $X_{i}$ are small, if max is small:

$$=P(X_{1} \le y *....X_{n} \le y)$$

they are independent:
$$=P(X_{1} \le y) *....(X_{n} \le y)$$

cdf of uniform distribution on $[0,\theta]$:

$$=(\frac{y}{0})^{n},  0\le y \le \theta$$

Now taking the derivative:

$$\frac{d}{df_{Y}}=
\begin{cases}
 \frac{ny^{n-1}}{\theta^{n}} & 0\le y \le \theta \\
 0 & \text{otherwise} 
\end{cases}$$ 




%%%%%%%%%%%%%%%%%%%%%%%%%%%%%%%%%%%%%%
\newpage

\textbf{32b)} Use the result of part (a) to show that mle is biased but that (n + 1)max$(X_{i})/n$ is unbiased.

For the estimator $(E(y)=\theta)$ to be unbiased, then: 

$$E(Y)=\int_{0}^{\theta} y * \frac{ny^n-1}{\theta^n} dy$$
$$=\frac{n}{\theta^n}\frac{y^n+1}{n+1}|_{0}^{\theta}$$
$$=\frac{n}{n+1}\theta$$

As you can see that $E(Y) \neq \theta$, and $\therefore$ the estimator is biased.

\vspace{3mm}

To show that (n + 1)max$(X_{i})/n$ is unbiased:

Let $\tilde{Y}=\frac{n+1}{n}$, then

$$E(\tilde{Y})=E\frac{(n+1)}{n}$$
Using result from (a):
$$=E(Y)\frac{n+1}{n}$$
$$=\theta \frac{n}{n+1}\frac{n+1}{n}$$

Cancel:
$$=\theta$$

This shows that $\tilde{Y}$ is \hl{unbiased}.





%%%%%%%%%%%%%%%%%%%%%%%%%%%%%%%%%%%%%%
\newpage

\textbf{36)} Compute both the corresponding point estimate and s for the data of Ex 6.2.

The Data from Ex 6.2:

\vspace{3mm}

$\begin{matrix}
24.46 & 25.61 & 26.25 & 26.42 & 26.66  & 27.15 & 27.31 & 27.54 & 27.74 & 27.94 \\
27.98 & 28.04 & 28.28 & 28.49 & 28.50 & 28.87 & 29.11 & 29.13 & 29.50 & 30.88
\end{matrix}$

\vspace{2mm}

To find S (Standard Deviation), we need to find $S^{2}$ (Sample Variance) first:

$$S^{2}=\frac{\sum(X_{i}-\bar{X})^2}{n-1}$$


$\bar{X}=X_{1}+X_{2}+...X_{20}= 555.86$

$X_{i}^{2}=24.46^{2}+5.61^{2}+....30.88^{2}=15489.62$

Now that we found the little parts we need to solve this problem, we can find our answer:


$$S^{2}=\frac{\sum(X_{i}-\bar{X})^2}{n-1}$$
$$=\frac{1}{n-1}(\sum x_{i}^{2} -\frac{1}{n}(\sum x_{i})^2)$$
$$=\frac{1}{20-1}(15489.62-\frac{1}{20}(555.86)^2)$$
$$=\frac{1}{20-1}(40.6034)$$
$$S^{2}=2.137$$

Now we can find Standard Deviation (S):

$$S=\sqrt{(S^{2})}$$
$$=\sqrt{(2.137)}$$
$$=1.462$$


So S=1.462

\vspace{2mm}
Now to find the estimate of standard deviation, we have to find median:

$\begin{matrix}
\xout{24.46} & \xout{25.61} & \xout{26.25} & \xout{26.42} & \xout{26.66}  & \xout{27.15} & \xout{27.31} & \xout{27.54} & \xout{27.74} & \textbf{27.94} \\
\textbf{27.98} & \xout{28.04} & \xout{28.28} & \xout{28.49} & \xout{28.50} & \xout{28.87} & \xout{29.11} & \xout{29.13} & \xout{29.50} & \xout{30.88}
\end{matrix}$
\vspace{2mm}

Median:$$\frac{27.94 + 27.98}{2}=27.96$$

Now we have too subtract our sample by the median and absolute the sample after that:
\vspace{2mm}

$\begin{matrix}
3.5 & 2.35 & 1.71 & 1.54 & 1.3  & 0.81 & 0.65 & 0.42 & 0.22 & 0.02\\
0.02 & 0.08 & 0.32 & 0.53 &0.54 & 0.91 & 1.15 & 1.17 & 1.54 & 2.92
\end{matrix}$

Now that we have this, we can now divide by .6745 (given in the problem) and put them in order:

$$3.5/.6745, 2.35/.6745......2.92/.6745$$
$$.$$
$$.$$
$$.$$
$$.$$

$\begin{matrix}
0.03 & 0.03 & 0.12 & 0.33 & 0.47  & 0.62 & 0.79 & 0.80 & 0.96 & 1.20\\
1.35 &1.70 & 1.73 & 1.93 & 2.28 & 2.28 & 2.54 & 3.48 & 4.33 &  5.19
\end{matrix}$

\vspace{2mm}

Now finding the median for this data: 
\vspace{2mm}

$\begin{matrix}
\xout{0.03} & \xout{0.03} & \xout{0.12} & \xout{0.33} & \xout{0.47 }  & \xout{0.62} & \xout{0.79} & \xout{0.80} & \xout{0.96 } & \textbf{1.20} \\
\textbf{1.35} & \xout{1.70} & \xout{1.73} & \xout{1.93} & \xout{2.28} &  \xout{2.28} & \xout{2.54} & \xout{3.48} & \xout{4.33} & \xout{5.19} 
\end{matrix}$
\vspace{2mm}


Because there are two number for the median, we have to add the two values and divid it by 2:

$$\frac{1.20 + 1.35} {2}=1.275$$

With everything being solved:

$$\hat{\sigma}=1.275$$

\begin{center}
\hl{Answers:} $S=1.462$ and $\hat{\sigma}=1.275$
\end{center}




  \end{document}

